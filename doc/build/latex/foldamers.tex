%% Generated by Sphinx.
\def\sphinxdocclass{report}
\documentclass[letterpaper,12pt,english,openany,oneside]{sphinxmanual}
\ifdefined\pdfpxdimen
   \let\sphinxpxdimen\pdfpxdimen\else\newdimen\sphinxpxdimen
\fi \sphinxpxdimen=.75bp\relax

\PassOptionsToPackage{warn}{textcomp}
\usepackage[utf8]{inputenc}
\ifdefined\DeclareUnicodeCharacter
% support both utf8 and utf8x syntaxes
\edef\sphinxdqmaybe{\ifdefined\DeclareUnicodeCharacterAsOptional\string"\fi}
  \DeclareUnicodeCharacter{\sphinxdqmaybe00A0}{\nobreakspace}
  \DeclareUnicodeCharacter{\sphinxdqmaybe2500}{\sphinxunichar{2500}}
  \DeclareUnicodeCharacter{\sphinxdqmaybe2502}{\sphinxunichar{2502}}
  \DeclareUnicodeCharacter{\sphinxdqmaybe2514}{\sphinxunichar{2514}}
  \DeclareUnicodeCharacter{\sphinxdqmaybe251C}{\sphinxunichar{251C}}
  \DeclareUnicodeCharacter{\sphinxdqmaybe2572}{\textbackslash}
\fi
\usepackage{cmap}
\usepackage[T1]{fontenc}
\usepackage{amsmath,amssymb,amstext}
\usepackage{babel}
\usepackage{times}
\usepackage[Bjarne]{fncychap}
\usepackage{sphinx}

\fvset{fontsize=\small}
\usepackage{geometry}

% Include hyperref last.
\usepackage{hyperref}
% Fix anchor placement for figures with captions.
\usepackage{hypcap}% it must be loaded after hyperref.
% Set up styles of URL: it should be placed after hyperref.
\urlstyle{same}

\addto\captionsenglish{\renewcommand{\figurename}{Fig.\@ }}
\makeatletter
\def\fnum@figure{\figurename\thefigure{}}
\makeatother
\addto\captionsenglish{\renewcommand{\tablename}{Table }}
\makeatletter
\def\fnum@table{\tablename\thetable{}}
\makeatother
\addto\captionsenglish{\renewcommand{\literalblockname}{Listing}}

\addto\captionsenglish{\renewcommand{\literalblockcontinuedname}{continued from previous page}}
\addto\captionsenglish{\renewcommand{\literalblockcontinuesname}{continues on next page}}
\addto\captionsenglish{\renewcommand{\sphinxnonalphabeticalgroupname}{Non-alphabetical}}
\addto\captionsenglish{\renewcommand{\sphinxsymbolsname}{Symbols}}
\addto\captionsenglish{\renewcommand{\sphinxnumbersname}{Numbers}}

\addto\extrasenglish{\def\pageautorefname{page}}

\setcounter{tocdepth}{1}



\title{foldamers Documentation}
\date{Jun 03, 2019}
\release{0.0}
\author{Garrett A. Meek\\Lenny T. Fobe\\ \\Research group of Professor Michael R. Shirts\\ \\Dept. of Chemical and Biological Engineering\\University of Colorado Boulder}
\newcommand{\sphinxlogo}{\vbox{}}
\renewcommand{\releasename}{Release}
\makeindex
\begin{document}

\pagestyle{empty}
\sphinxmaketitle
\pagestyle{plain}
\sphinxtableofcontents
\pagestyle{normal}
\phantomsection\label{\detokenize{index::doc}}


This documentation is generated automatically using Sphinx, which reads all docstring-formatted comments from Python functions in the ‘foldamers’ repository.  (See foldamers/doc for Sphinx source files.)


\chapter{Coarse grained model utilities}
\label{\detokenize{cg_model:coarse-grained-model-utilities}}\label{\detokenize{cg_model::doc}}
This page details the functions and classes in src/cg\_model/cgmodel.py


\section{The ‘basic\_cgmodel’ function to build coarse grained oligomers}
\label{\detokenize{cg_model:the-basic-cgmodel-function-to-build-coarse-grained-oligomers}}
Shown below is the ‘basic\_cgmodel’ function, which requires only a minimal set of input arguments to build a coarse grained model.  Given a set of input arguments this function creates a CGModel() class object, applying a set of default values for un-defined parameters.

\phantomsection\label{\detokenize{cg_model:module-cg_model.cgmodel}}\index{cg\_model.cgmodel (module)@\spxentry{cg\_model.cgmodel}\spxextra{module}}\index{basic\_cgmodel() (in module cg\_model.cgmodel)@\spxentry{basic\_cgmodel()}\spxextra{in module cg\_model.cgmodel}}

\begin{fulllineitems}
\phantomsection\label{\detokenize{cg_model:cg_model.cgmodel.basic_cgmodel}}\pysiglinewithargsret{\sphinxcode{\sphinxupquote{cg\_model.cgmodel.}}\sphinxbfcode{\sphinxupquote{basic\_cgmodel}}}{\emph{polymer\_length=8, backbone\_length=1, sidechain\_length=1, sidechain\_positions={[}0{]}, mass=Quantity(value=12.0, unit=dalton), bond\_length=Quantity(value=1.0, unit=angstrom), sigma=Quantity(value=2.5, unit=angstrom), epsilon=Quantity(value=0.5, unit=kilocalorie/mole), positions=None}}{}~\begin{quote}\begin{description}
\item[{Parameters}] \leavevmode\begin{itemize}
\item {} 
\sphinxstyleliteralstrong{\sphinxupquote{polymer\_length}} (\sphinxstyleliteralemphasis{\sphinxupquote{integer}}) \textendash{} Number of monomer units, default = 8

\item {} 
\sphinxstyleliteralstrong{\sphinxupquote{backbone\_length}} (\sphinxstyleliteralemphasis{\sphinxupquote{integer}}) \textendash{} Defines the number of beads in the backbone (assumes all monomers have the same backbone length), default = 1

\item {} 
\sphinxstyleliteralstrong{\sphinxupquote{sidechain\_length}} (\sphinxstyleliteralemphasis{\sphinxupquote{integer}}) \textendash{} Defines the number of beads in the sidechain ( assumes all monomers have the same sidechain length), default = 1

\item {} 
\sphinxstyleliteralstrong{\sphinxupquote{sidechain\_positions}} (\sphinxstyleliteralemphasis{\sphinxupquote{List}}\sphinxstyleliteralemphasis{\sphinxupquote{( }}\sphinxstyleliteralemphasis{\sphinxupquote{integer}}\sphinxstyleliteralemphasis{\sphinxupquote{ )}}) \textendash{} Defines the backbone bead indices upon which we will place the sidechains, default = {[}0{]}

\item {} 
\sphinxstyleliteralstrong{\sphinxupquote{mass}} (\sphinxstyleliteralemphasis{\sphinxupquote{float * simtk.unit}}) \textendash{} Mass for all coarse grained beads, default = 12.0 * unit.amu

\item {} 
\sphinxstyleliteralstrong{\sphinxupquote{bond\_length}} (\sphinxstyleliteralemphasis{\sphinxupquote{float * simtk.unit}}) \textendash{} Defines the length for all bond types, default = 1.0 * unit.angstrom

\item {} 
\sphinxstyleliteralstrong{\sphinxupquote{sigma}} (\sphinxstyleliteralemphasis{\sphinxupquote{float * simtk.unit}}) \textendash{} Non-bonded bead Lennard-Jones equilibrium interaction distance, default = 2.5 * bond\_length (for all particle interactions)

\item {} 
\sphinxstyleliteralstrong{\sphinxupquote{epsilon}} \textendash{} Non-bonded Lennard-Jones equilibrium interaction energy, default = 0.5 * unit.kilocalorie\_per\_mole

\item {} 
\sphinxstyleliteralstrong{\sphinxupquote{positions}} (\sphinxstyleliteralemphasis{\sphinxupquote{np.array}}\sphinxstyleliteralemphasis{\sphinxupquote{( }}\sphinxstyleliteralemphasis{\sphinxupquote{float * simtk.unit}}\sphinxstyleliteralemphasis{\sphinxupquote{ ( }}\sphinxstyleliteralemphasis{\sphinxupquote{shape = num\_beads x 3}}\sphinxstyleliteralemphasis{\sphinxupquote{ ) }}\sphinxstyleliteralemphasis{\sphinxupquote{)}}) \textendash{} Positions for coarse grained particles in the model, default = None

\end{itemize}

\end{description}\end{quote}

cgmodel: CGModel() class object

\end{fulllineitems}



\section{Full ‘CGModel’ class to build/model coarse grained oligomers}
\label{\detokenize{cg_model:full-cgmodel-class-to-build-model-coarse-grained-oligomers}}
Shown below is a detailed description of the full ‘cgmodel’ class object.

\phantomsection\label{\detokenize{cg_model:module-cg_model.cgmodel}}\index{cg\_model.cgmodel (module)@\spxentry{cg\_model.cgmodel}\spxextra{module}}\index{CGModel (class in cg\_model.cgmodel)@\spxentry{CGModel}\spxextra{class in cg\_model.cgmodel}}

\begin{fulllineitems}
\phantomsection\label{\detokenize{cg_model:cg_model.cgmodel.CGModel}}\pysiglinewithargsret{\sphinxbfcode{\sphinxupquote{class }}\sphinxcode{\sphinxupquote{cg\_model.cgmodel.}}\sphinxbfcode{\sphinxupquote{CGModel}}}{\emph{positions=None, polymer\_length=8, backbone\_lengths={[}1{]}, sidechain\_lengths={[}1{]}, sidechain\_positions={[}0{]}, masses=\{'backbone\_bead\_masses': Quantity(value=10.0, unit=dalton), 'sidechain\_bead\_masses': Quantity(value=10.0, unit=dalton)\}, sigmas=\{'bb\_bb\_sigma': Quantity(value=2.5, unit=angstrom), 'bb\_sc\_sigma': Quantity(value=2.5, unit=angstrom), 'sc\_sc\_sigma': Quantity(value=2.5, unit=angstrom)\}, epsilons=\{'bb\_bb\_eps': Quantity(value=0.5, unit=kilocalorie/mole), 'bb\_sc\_eps': Quantity(value=0.5, unit=kilocalorie/mole), 'sc\_sc\_eps': Quantity(value=0.5, unit=kilocalorie/mole)\}, bond\_lengths=\{'bb\_bb\_bond\_length': Quantity(value=1.0, unit=angstrom), 'bb\_sc\_bond\_length': Quantity(value=1.0, unit=angstrom), 'sc\_sc\_bond\_length': Quantity(value=1.0, unit=angstrom)\}, bond\_force\_constants=None, bond\_angle\_force\_constants=None, torsion\_force\_constants=None, equil\_bond\_angle=None, equil\_dihedral\_angle=None, charges=None, constrain\_bonds=True, include\_bond\_forces=True, include\_nonbonded\_forces=True, include\_bond\_angle\_forces=True, include\_torsion\_forces=True, check\_energy\_conservation=True, homopolymer=True}}{}~\begin{quote}\begin{description}
\item[{Parameters}] \leavevmode\begin{itemize}
\item {} 
\sphinxstyleliteralstrong{\sphinxupquote{positions}} (\sphinxstyleliteralemphasis{\sphinxupquote{np.array}}\sphinxstyleliteralemphasis{\sphinxupquote{( }}\sphinxstyleliteralemphasis{\sphinxupquote{float * simtk.unit}}\sphinxstyleliteralemphasis{\sphinxupquote{ ( }}\sphinxstyleliteralemphasis{\sphinxupquote{shape = num\_beads x 3}}\sphinxstyleliteralemphasis{\sphinxupquote{ ) }}\sphinxstyleliteralemphasis{\sphinxupquote{)}}) \textendash{} Positions for all of the particles, default = None

\item {} 
\sphinxstyleliteralstrong{\sphinxupquote{polymer\_length}} (\sphinxstyleliteralemphasis{\sphinxupquote{integer}}) \textendash{} Length of the polymer, default = 8

\item {} 
\sphinxstyleliteralstrong{\sphinxupquote{backbone\_lengths}} (\sphinxstyleliteralemphasis{\sphinxupquote{List}}\sphinxstyleliteralemphasis{\sphinxupquote{( }}\sphinxstyleliteralemphasis{\sphinxupquote{integer}}\sphinxstyleliteralemphasis{\sphinxupquote{ )}}) \textendash{} Defines the number of beads in the backbone for each monomer type, default = {[}1{]}

\item {} 
\sphinxstyleliteralstrong{\sphinxupquote{sidechain\_lengths}} (\sphinxstyleliteralemphasis{\sphinxupquote{List}}\sphinxstyleliteralemphasis{\sphinxupquote{( }}\sphinxstyleliteralemphasis{\sphinxupquote{integer}}\sphinxstyleliteralemphasis{\sphinxupquote{ )}}) \textendash{} Defines the number of beads in the sidechain for each monomer type, default = {[}1{]}

\item {} 
\sphinxstyleliteralstrong{\sphinxupquote{sidechain\_positions}} (\sphinxstyleliteralemphasis{\sphinxupquote{List}}\sphinxstyleliteralemphasis{\sphinxupquote{( }}\sphinxstyleliteralemphasis{\sphinxupquote{integer}}\sphinxstyleliteralemphasis{\sphinxupquote{ )}}) \textendash{} Defines the backbone bead indices where sidechains are positioned, default = {[}0{]} (Place a sidechain on the first backbone bead in each monomer.)

\item {} 
\sphinxstyleliteralstrong{\sphinxupquote{masses}} (\sphinxhref{https://docs.python.org/3/library/stdtypes.html\#dict}{\sphinxstyleliteralemphasis{\sphinxupquote{dict}}}\sphinxstyleliteralemphasis{\sphinxupquote{( }}\sphinxstyleliteralemphasis{\sphinxupquote{'backbone\_bead\_masses': float * simtk.unit}}\sphinxstyleliteralemphasis{\sphinxupquote{, }}\sphinxstyleliteralemphasis{\sphinxupquote{'sidechain\_bead\_masses': float * simtk.unit}}\sphinxstyleliteralemphasis{\sphinxupquote{ )}}) \textendash{} Masses of all particle types, default = 10.0 * unit.amu (for all particles)

\item {} 
\sphinxstyleliteralstrong{\sphinxupquote{sigmas}} (\sphinxhref{https://docs.python.org/3/library/stdtypes.html\#dict}{\sphinxstyleliteralemphasis{\sphinxupquote{dict}}}\sphinxstyleliteralemphasis{\sphinxupquote{( }}\sphinxstyleliteralemphasis{\sphinxupquote{'bb\_bb\_sigma': float * simtk.unit}}\sphinxstyleliteralemphasis{\sphinxupquote{,}}\sphinxstyleliteralemphasis{\sphinxupquote{'bb\_sc\_sigma': float * simtk.unit}}\sphinxstyleliteralemphasis{\sphinxupquote{,}}\sphinxstyleliteralemphasis{\sphinxupquote{'sc\_sc\_sigma': float * simtk.unit\}}}) \textendash{} Non-bonded bead Lennard-Jones equilibrium interaction distances, default = 2.5 unit.angstrom (for all particles)

\item {} 
\sphinxstyleliteralstrong{\sphinxupquote{epsilons}} (\sphinxhref{https://docs.python.org/3/library/stdtypes.html\#dict}{\sphinxstyleliteralemphasis{\sphinxupquote{dict}}}\sphinxstyleliteralemphasis{\sphinxupquote{( }}\sphinxstyleliteralemphasis{\sphinxupquote{'bb\_bb\_eps': float * simtk.unit}}\sphinxstyleliteralemphasis{\sphinxupquote{,}}\sphinxstyleliteralemphasis{\sphinxupquote{'bb\_sc\_eps': float * simtk.unit}}\sphinxstyleliteralemphasis{\sphinxupquote{,}}\sphinxstyleliteralemphasis{\sphinxupquote{'sc\_sc\_eps': float * simtk.unit}}\sphinxstyleliteralemphasis{\sphinxupquote{ )}}) \textendash{} Non-bonded Lennard-Jones equilibrium interaction strengths, default = 0.5 * unit.kilocalorie\_per\_mole (for all particle interactions types)

\item {} 
\sphinxstyleliteralstrong{\sphinxupquote{bond\_lengths}} (\sphinxhref{https://docs.python.org/3/library/stdtypes.html\#dict}{\sphinxstyleliteralemphasis{\sphinxupquote{dict}}}\sphinxstyleliteralemphasis{\sphinxupquote{( }}\sphinxstyleliteralemphasis{\sphinxupquote{'bb\_bb\_bond\_length': float * simtk.unit}}\sphinxstyleliteralemphasis{\sphinxupquote{,}}\sphinxstyleliteralemphasis{\sphinxupquote{'bb\_sc\_bond\_length': float * simtk.unit}}\sphinxstyleliteralemphasis{\sphinxupquote{,}}\sphinxstyleliteralemphasis{\sphinxupquote{'sc\_sc\_bond\_length': float * simtk.unit}}\sphinxstyleliteralemphasis{\sphinxupquote{ )}}) \textendash{} Bond lengths for all bonds, default = 1.0 unit.angstrom

\item {} 
\sphinxstyleliteralstrong{\sphinxupquote{bond\_force\_constants}} (\sphinxhref{https://docs.python.org/3/library/stdtypes.html\#dict}{\sphinxstyleliteralemphasis{\sphinxupquote{dict}}}\sphinxstyleliteralemphasis{\sphinxupquote{( }}\sphinxstyleliteralemphasis{\sphinxupquote{'bb\_bb\_bond\_k': float}}\sphinxstyleliteralemphasis{\sphinxupquote{,}}\sphinxstyleliteralemphasis{\sphinxupquote{'bb\_sc\_bond\_k': float}}\sphinxstyleliteralemphasis{\sphinxupquote{, }}\sphinxstyleliteralemphasis{\sphinxupquote{'sc\_sc\_bond\_k': float}}\sphinxstyleliteralemphasis{\sphinxupquote{ )}}) \textendash{} Bond force constants for all bond types, default = 9.9e9 ( implied units are: kJ/mol/A\textasciicircum{}2 )

\item {} 
\sphinxstyleliteralstrong{\sphinxupquote{charges}} (\sphinxhref{https://docs.python.org/3/library/stdtypes.html\#dict}{\sphinxstyleliteralemphasis{\sphinxupquote{dict}}}\sphinxstyleliteralemphasis{\sphinxupquote{( }}\sphinxstyleliteralemphasis{\sphinxupquote{'backbone\_bead\_charges': float * simtk.unit}}\sphinxstyleliteralemphasis{\sphinxupquote{,}}\sphinxstyleliteralemphasis{\sphinxupquote{'sidechain\_bead\_charges': float * simtk.unit}}\sphinxstyleliteralemphasis{\sphinxupquote{ )}}) \textendash{} Charges for all particles, default = 0.0 (for all particles)

\item {} 
\sphinxstyleliteralstrong{\sphinxupquote{num\_beads}} (\sphinxstyleliteralemphasis{\sphinxupquote{integer}}) \textendash{} Total number of particles in the coarse grained model, default = 16 (The total number of particles in a length=8 1-1 coarse-grained model)

\item {} 
\sphinxstyleliteralstrong{\sphinxupquote{system}} (\sphinxstyleliteralemphasis{\sphinxupquote{OpenMM System}}\sphinxstyleliteralemphasis{\sphinxupquote{(}}\sphinxstyleliteralemphasis{\sphinxupquote{) }}\sphinxstyleliteralemphasis{\sphinxupquote{class object}}) \textendash{} OpenMM System() object, which stores the forces for the coarse grained model, default = None

\item {} 
\sphinxstyleliteralstrong{\sphinxupquote{topology}} (\sphinxstyleliteralemphasis{\sphinxupquote{OpenMM Topology}}\sphinxstyleliteralemphasis{\sphinxupquote{(}}\sphinxstyleliteralemphasis{\sphinxupquote{) }}\sphinxstyleliteralemphasis{\sphinxupquote{class object}}) \textendash{} OpenMM Topology() object, which stores bonds, angles, and other structural attributes of the coarse grained model, default = None

\item {} 
\sphinxstyleliteralstrong{\sphinxupquote{constrain\_bonds}} (\sphinxstyleliteralemphasis{\sphinxupquote{Logical}}) \textendash{} Logical variable determining whether bond constraints are applied during a simulation of the energy for the system, default = True

\item {} 
\sphinxstyleliteralstrong{\sphinxupquote{include\_bond\_forces}} (\sphinxstyleliteralemphasis{\sphinxupquote{Logical}}) \textendash{} Include contributions from bond potentials when calculating the potential energy, default = True

\item {} 
\sphinxstyleliteralstrong{\sphinxupquote{include\_nonbonded\_forces}} (\sphinxstyleliteralemphasis{\sphinxupquote{Logical}}) \textendash{} Include contributions from nonbonded interactions when calculating the potential energy, default = True

\item {} 
\sphinxstyleliteralstrong{\sphinxupquote{include\_bond\_angle\_forces}} (\sphinxstyleliteralemphasis{\sphinxupquote{Logical}}) \textendash{} Include contributions from bond angle forces when calculating the potential energy, default = True

\item {} 
\sphinxstyleliteralstrong{\sphinxupquote{include\_torsion\_forces}} (\sphinxstyleliteralemphasis{\sphinxupquote{Logical}}) \textendash{} Include contributions from torsions when calculating the potential energy, default = True

\end{itemize}

\end{description}\end{quote}

\sphinxstylestrong{Attributes:}
\begin{description}
\item[{polymer\_length}] \leavevmode{[}integer{]}
Returns the number of monomers in the polymer/oligomer

\item[{backbone\_lengths}] \leavevmode{[}List( integers ){]}
Returns a list of all unique backbone legnths (for individual monomers) in this model

\item[{sidechain\_lengths}] \leavevmode{[}List( integers ){]}
Returns a list of all unique sidechain lengths (for individual monomers) in this model

\item[{sidechain\_positions}] \leavevmode{[}List( integers ){]}
Returns a list of integers for all uniqye sidechain positions (along the backbone, for individual monomers) in this model

\item[{masses}] \leavevmode{[}dict( float * simtk.unit ){]}
Returns a list of the particle masses for all unique particle definitions in this model

\item[{sigmas}] \leavevmode{[}dict( float * simtk.unit ){]}
Returns a list of the Lennard-Jones nonbonded interaction distances for all unique particle interaction types

\item[{epsilons}] \leavevmode{[}dict ( float * simtk.unit ){]}
Returns a list of the Lennard-Jones nonbonded interaction strengths (well-depth) for all unique particle interaction types

\item[{bond\_lengths}] \leavevmode{[}dict ( float * simtk.unit ){]}
Returns a list of the bond lengths for all unique bond definitions in the model

\item[{nonbonded\_interaction\_list}] \leavevmode{[}List( List( integer, integer ) ){]}
Returns a list of the indices for particles that exhibit nonbonded interactions in this model

\item[{bond\_list}] \leavevmode{[}List( List( integer, integer ) ){]}
Returns a list of paired indices for particles that are bonded in this model

\item[{bond\_angle\_list}] \leavevmode{[}List( List( integer, integer, integer ) ){]}
Returns a unique list of indices for all combinations of three particles that share a set of two bonds

\item[{torsion\_list: List( List( integer, integer, integer, integer ) )}] \leavevmode
Returns a unique list of indices for all ocmbinations of four particles that define a torsion (minimum requirement is that they share a set of three bonds)

\item[{bond\_force\_constants}] \leavevmode{[}Dict( float ){]}
Returns a dictionary with definitions for the bond force constants for all unique bond definitions

\item[{bond\_angle\_force\_constants: Dict( float )}] \leavevmode
Returns a dictionary with definitions for the bond angle force constants for all unique bond angle definitions

\item[{torsion\_force\_constants: Dict( float )}] \leavevmode
Returns a dictionary with definitions for the torsion force constants for all unique torsion definitions

\item[{equil\_dihedral\_angle}] \leavevmode{[}Dict( float ){]}
Returns the equilibrium dihedral angle for all unique torsion definitions

\item[{charges}] \leavevmode{[}Dict( float * simtk.unit ){]}
Returns the charges for all unique particle definitions in this model

\item[{num\_beads}] \leavevmode{[}integer{]}
Returns the number of particles in this model

\item[{positions}] \leavevmode{[}np.array( float * simtk.unit ( shape = num\_beads x 3 ) ){]}
Returns the currently-stored positions for this model (if any)

\item[{system}] \leavevmode{[}System() class object{]}
Returns the currently-stored OpenMM System() object for this model (if any)

\item[{topology}] \leavevmode{[}Topology() class object{]}
Returns the currently-stored OpenMM Topology() object for this model (if any)

\item[{constrain\_bonds}] \leavevmode{[}Logical{]}
Returns the current setting for bond constraints in the model

\item[{include\_bond\_forces}] \leavevmode{[}Logical{]}
Indicates if bond forces are currently included when calculating the energy

\item[{include\_nonbonded\_forces}] \leavevmode{[}Logical{]}
Indicates if nonbonded interactions are currently included when calculating the energy

\item[{include\_bond\_angle\_forces}] \leavevmode{[}Logical{]}
Indicates if bond angle forces are currently included when calculating the energy

\item[{include\_torsion\_forces}] \leavevmode{[}Logical{]}
Indicates if torsion potentials are currently included when calculating the energy

\end{description}
\index{check\_energy\_conservation (cg\_model.cgmodel.CGModel attribute)@\spxentry{check\_energy\_conservation}\spxextra{cg\_model.cgmodel.CGModel attribute}}

\begin{fulllineitems}
\phantomsection\label{\detokenize{cg_model:cg_model.cgmodel.CGModel.check_energy_conservation}}\pysigline{\sphinxbfcode{\sphinxupquote{check\_energy\_conservation}}\sphinxbfcode{\sphinxupquote{ = None}}}
Get bond, angle, and torsion lists.

\end{fulllineitems}

\index{constrain\_bonds (cg\_model.cgmodel.CGModel attribute)@\spxentry{constrain\_bonds}\spxextra{cg\_model.cgmodel.CGModel attribute}}

\begin{fulllineitems}
\phantomsection\label{\detokenize{cg_model:cg_model.cgmodel.CGModel.constrain_bonds}}\pysigline{\sphinxbfcode{\sphinxupquote{constrain\_bonds}}\sphinxbfcode{\sphinxupquote{ = None}}}
Make a list of coarse grained particle masses:

\end{fulllineitems}

\index{get\_all\_particle\_masses() (cg\_model.cgmodel.CGModel method)@\spxentry{get\_all\_particle\_masses()}\spxextra{cg\_model.cgmodel.CGModel method}}

\begin{fulllineitems}
\phantomsection\label{\detokenize{cg_model:cg_model.cgmodel.CGModel.get_all_particle_masses}}\pysiglinewithargsret{\sphinxbfcode{\sphinxupquote{get\_all\_particle\_masses}}}{}{}
Returns a list of unique particle masses

self: CGModel() class object

List( unique particle masses )

\end{fulllineitems}

\index{get\_bond\_angle() (cg\_model.cgmodel.CGModel method)@\spxentry{get\_bond\_angle()}\spxextra{cg\_model.cgmodel.CGModel method}}

\begin{fulllineitems}
\phantomsection\label{\detokenize{cg_model:cg_model.cgmodel.CGModel.get_bond_angle}}\pysiglinewithargsret{\sphinxbfcode{\sphinxupquote{get\_bond\_angle}}}{\emph{particle\_1\_index}, \emph{particle\_2\_index}, \emph{particle\_3\_index}}{}
Determines the correct equilibrium bond angle between three particles

self: CGModel() class object

particle\_1\_index: Index of the first particle in the bond, default = None

particle\_2\_index: Index of the second particle in the bond angle, default = None

particle\_3\_index: Index of the third particle in the bond angle, default = None

bond\_angle: Bond angle for the two bonds defined by these three particles.

\end{fulllineitems}

\index{get\_bond\_angle\_force\_constant() (cg\_model.cgmodel.CGModel method)@\spxentry{get\_bond\_angle\_force\_constant()}\spxextra{cg\_model.cgmodel.CGModel method}}

\begin{fulllineitems}
\phantomsection\label{\detokenize{cg_model:cg_model.cgmodel.CGModel.get_bond_angle_force_constant}}\pysiglinewithargsret{\sphinxbfcode{\sphinxupquote{get\_bond\_angle\_force\_constant}}}{\emph{particle\_1\_index}, \emph{particle\_2\_index}, \emph{particle\_3\_index}}{}
Determines the correct equilibrium bond angle between three particles

self: CGModel() class object

particle\_1\_index: Index of the first particle in the bond, default = None

particle\_2\_index: Index of the second particle in the bond angle, default = None

particle\_3\_index: Index of the third particle in the bond angle, default = None

bond\_angle: Bond angle for the two bonds defined by these three particles.

\end{fulllineitems}

\index{get\_bond\_angle\_list() (cg\_model.cgmodel.CGModel method)@\spxentry{get\_bond\_angle\_list()}\spxextra{cg\_model.cgmodel.CGModel method}}

\begin{fulllineitems}
\phantomsection\label{\detokenize{cg_model:cg_model.cgmodel.CGModel.get_bond_angle_list}}\pysiglinewithargsret{\sphinxbfcode{\sphinxupquote{get\_bond\_angle\_list}}}{}{}
Construct a list of indices for particles that define bond angles in our coarse grained model

\end{fulllineitems}

\index{get\_bond\_force\_constant() (cg\_model.cgmodel.CGModel method)@\spxentry{get\_bond\_force\_constant()}\spxextra{cg\_model.cgmodel.CGModel method}}

\begin{fulllineitems}
\phantomsection\label{\detokenize{cg_model:cg_model.cgmodel.CGModel.get_bond_force_constant}}\pysiglinewithargsret{\sphinxbfcode{\sphinxupquote{get\_bond\_force\_constant}}}{\emph{particle\_1\_index}, \emph{particle\_2\_index}}{}
Determines the correct bond force constant for two particles

cgmodel: CGModel() class object

particle\_1\_index: Index of the first particle in the bond, default = None

particle\_2\_index: Index of the second particle in the bond, default = None

bond\_force\_constant: Bond force constant for the bond defined by these two particles

\end{fulllineitems}

\index{get\_bond\_length() (cg\_model.cgmodel.CGModel method)@\spxentry{get\_bond\_length()}\spxextra{cg\_model.cgmodel.CGModel method}}

\begin{fulllineitems}
\phantomsection\label{\detokenize{cg_model:cg_model.cgmodel.CGModel.get_bond_length}}\pysiglinewithargsret{\sphinxbfcode{\sphinxupquote{get\_bond\_length}}}{\emph{particle\_1\_index}, \emph{particle\_2\_index}}{}
Determines the correct bond force constant for two particles

self: CGModel() class object

particle\_1\_index: Index of the first particle in the bond
( integer )
Default = None

particle\_2\_index: Index of the second particle in the bond
( integer )
Default = None

bond\_length: Bond length for the bond defined by these two particles.
( simtk.unit.Quantity() )

\end{fulllineitems}

\index{get\_bond\_length\_from\_names() (cg\_model.cgmodel.CGModel method)@\spxentry{get\_bond\_length\_from\_names()}\spxextra{cg\_model.cgmodel.CGModel method}}

\begin{fulllineitems}
\phantomsection\label{\detokenize{cg_model:cg_model.cgmodel.CGModel.get_bond_length_from_names}}\pysiglinewithargsret{\sphinxbfcode{\sphinxupquote{get\_bond\_length\_from\_names}}}{\emph{particle\_1\_name}, \emph{particle\_2\_name}}{}
Determines the correct bond length for two particles, given their symbols.

cgmodel: CGModel() class object

particle\_1\_name: Symbol for the first particle in the bond
( string )
Default = None

particle\_2\_name: Symbol for the second particle in the bond
( string )
Default = None

bond\_length: Bond length for the bond defined by these two particles.
( simtk.unit.Quantity() )

\end{fulllineitems}

\index{get\_bond\_list() (cg\_model.cgmodel.CGModel method)@\spxentry{get\_bond\_list()}\spxextra{cg\_model.cgmodel.CGModel method}}

\begin{fulllineitems}
\phantomsection\label{\detokenize{cg_model:cg_model.cgmodel.CGModel.get_bond_list}}\pysiglinewithargsret{\sphinxbfcode{\sphinxupquote{get\_bond\_list}}}{}{}
Construct a bond list for the coarse grained model

\end{fulllineitems}

\index{get\_epsilon() (cg\_model.cgmodel.CGModel method)@\spxentry{get\_epsilon()}\spxextra{cg\_model.cgmodel.CGModel method}}

\begin{fulllineitems}
\phantomsection\label{\detokenize{cg_model:cg_model.cgmodel.CGModel.get_epsilon}}\pysiglinewithargsret{\sphinxbfcode{\sphinxupquote{get\_epsilon}}}{\emph{particle\_index}, \emph{particle\_type=None}}{}
Returns the epsilon value for a particle, given its index.

self: CGModel() class object

Epsilon

\end{fulllineitems}

\index{get\_monomer\_types() (cg\_model.cgmodel.CGModel method)@\spxentry{get\_monomer\_types()}\spxextra{cg\_model.cgmodel.CGModel method}}

\begin{fulllineitems}
\phantomsection\label{\detokenize{cg_model:cg_model.cgmodel.CGModel.get_monomer_types}}\pysiglinewithargsret{\sphinxbfcode{\sphinxupquote{get\_monomer\_types}}}{}{}
Get a list of monomer dictionary objects for each unique monomer type.

\end{fulllineitems}

\index{get\_nonbonded\_interaction\_list() (cg\_model.cgmodel.CGModel method)@\spxentry{get\_nonbonded\_interaction\_list()}\spxextra{cg\_model.cgmodel.CGModel method}}

\begin{fulllineitems}
\phantomsection\label{\detokenize{cg_model:cg_model.cgmodel.CGModel.get_nonbonded_interaction_list}}\pysiglinewithargsret{\sphinxbfcode{\sphinxupquote{get\_nonbonded\_interaction\_list}}}{}{}
Construct a nonbonded interaction list for our coarse grained model

\end{fulllineitems}

\index{get\_num\_beads() (cg\_model.cgmodel.CGModel method)@\spxentry{get\_num\_beads()}\spxextra{cg\_model.cgmodel.CGModel method}}

\begin{fulllineitems}
\phantomsection\label{\detokenize{cg_model:cg_model.cgmodel.CGModel.get_num_beads}}\pysiglinewithargsret{\sphinxbfcode{\sphinxupquote{get\_num\_beads}}}{}{}
Calculate the number of beads in our coarse grained model(s)

\end{fulllineitems}

\index{get\_particle\_charge() (cg\_model.cgmodel.CGModel method)@\spxentry{get\_particle\_charge()}\spxextra{cg\_model.cgmodel.CGModel method}}

\begin{fulllineitems}
\phantomsection\label{\detokenize{cg_model:cg_model.cgmodel.CGModel.get_particle_charge}}\pysiglinewithargsret{\sphinxbfcode{\sphinxupquote{get\_particle\_charge}}}{\emph{particle\_index}}{}
Returns the charge for a particle, given its index.

self: CGModel() class object

Charge

\end{fulllineitems}

\index{get\_particle\_list() (cg\_model.cgmodel.CGModel method)@\spxentry{get\_particle\_list()}\spxextra{cg\_model.cgmodel.CGModel method}}

\begin{fulllineitems}
\phantomsection\label{\detokenize{cg_model:cg_model.cgmodel.CGModel.get_particle_list}}\pysiglinewithargsret{\sphinxbfcode{\sphinxupquote{get\_particle\_list}}}{}{}
Get a list of particles, where the indices correspond to those used in our system/topology

\end{fulllineitems}

\index{get\_particle\_mass() (cg\_model.cgmodel.CGModel method)@\spxentry{get\_particle\_mass()}\spxextra{cg\_model.cgmodel.CGModel method}}

\begin{fulllineitems}
\phantomsection\label{\detokenize{cg_model:cg_model.cgmodel.CGModel.get_particle_mass}}\pysiglinewithargsret{\sphinxbfcode{\sphinxupquote{get\_particle\_mass}}}{\emph{particle\_index}}{}
Returns the mass for a particle, given its index.

self: CGModel() class object

Mass

\end{fulllineitems}

\index{get\_particle\_type() (cg\_model.cgmodel.CGModel method)@\spxentry{get\_particle\_type()}\spxextra{cg\_model.cgmodel.CGModel method}}

\begin{fulllineitems}
\phantomsection\label{\detokenize{cg_model:cg_model.cgmodel.CGModel.get_particle_type}}\pysiglinewithargsret{\sphinxbfcode{\sphinxupquote{get\_particle\_type}}}{\emph{particle\_index}, \emph{particle\_name=None}}{}
Returns the name of a particle, given its index within the model

self: CGModel() class object

particle\_index: Index of the particle for which we would like to determine the type
Type: int()

particle\_type: ‘backbone’ or ‘sidechain’
Type: str()

\end{fulllineitems}

\index{get\_sigma() (cg\_model.cgmodel.CGModel method)@\spxentry{get\_sigma()}\spxextra{cg\_model.cgmodel.CGModel method}}

\begin{fulllineitems}
\phantomsection\label{\detokenize{cg_model:cg_model.cgmodel.CGModel.get_sigma}}\pysiglinewithargsret{\sphinxbfcode{\sphinxupquote{get\_sigma}}}{\emph{particle\_index}, \emph{particle\_type=None}}{}
Returns the sigma value for a particle, given its index within the coarse grained model.

self: CGModel() class object

Sigma

\end{fulllineitems}

\index{get\_torsion\_force\_constant() (cg\_model.cgmodel.CGModel method)@\spxentry{get\_torsion\_force\_constant()}\spxextra{cg\_model.cgmodel.CGModel method}}

\begin{fulllineitems}
\phantomsection\label{\detokenize{cg_model:cg_model.cgmodel.CGModel.get_torsion_force_constant}}\pysiglinewithargsret{\sphinxbfcode{\sphinxupquote{get\_torsion\_force\_constant}}}{\emph{torsion}}{}
Determines the torsion force constant given a list of particle indices

cgmodel: CGModel() class object

torsion: Indices of the particles in the torsion
( integer )
Default = None

torsion\_force\_constant: Force constant for the torsion defined by the input particles.
( Integer )

\end{fulllineitems}

\index{get\_torsion\_list() (cg\_model.cgmodel.CGModel method)@\spxentry{get\_torsion\_list()}\spxextra{cg\_model.cgmodel.CGModel method}}

\begin{fulllineitems}
\phantomsection\label{\detokenize{cg_model:cg_model.cgmodel.CGModel.get_torsion_list}}\pysiglinewithargsret{\sphinxbfcode{\sphinxupquote{get\_torsion\_list}}}{}{}
Construct a torsion list for our coarse grained model

\end{fulllineitems}


\end{fulllineitems}



\section{Other coarse grained model utilities}
\label{\detokenize{cg_model:module-cg_model.cgmodel}}\label{\detokenize{cg_model:other-coarse-grained-model-utilities}}\index{cg\_model.cgmodel (module)@\spxentry{cg\_model.cgmodel}\spxextra{module}}\index{get\_parent\_bead() (in module cg\_model.cgmodel)@\spxentry{get\_parent\_bead()}\spxextra{in module cg\_model.cgmodel}}

\begin{fulllineitems}
\phantomsection\label{\detokenize{cg_model:cg_model.cgmodel.get_parent_bead}}\pysiglinewithargsret{\sphinxcode{\sphinxupquote{cg\_model.cgmodel.}}\sphinxbfcode{\sphinxupquote{get\_parent\_bead}}}{\emph{cgmodel}, \emph{monomer\_index}, \emph{bead\_index}, \emph{backbone\_bead\_index=None}, \emph{sidechain\_bead=False}}{}
Determines the particle to which a given particle is bonded.  (Used for coarse grained model construction.)

cgmodel: CGModel() class object

monomer\_index: Index of the monomer the child particle belongs to.
( integer )
Default = None

bead\_index: Index of the particle for which we would like to determine the parent particle it is bonded to.
( integer )
Default = None

backbone\_bead\_index: If this bead is a backbone bead, this index tells us its index (within a monomer) along the backbone
( integer )
Default = None

sidechain\_bead: Logical variable stating whether or not this bead is in the sidechain.
( Logical )
Default = False

parent\_bead: Index for the particle that ‘bead\_index’ is bonded to.
( Integer )

\end{fulllineitems}



\chapter{Thermodynamic analysis tools for coarse grained modeling}
\label{\detokenize{thermo:thermodynamic-analysis-tools-for-coarse-grained-modeling}}\label{\detokenize{thermo::doc}}
This page details the functions and classes in src/thermo


\section{Tools to calculate the heat capacity with pymbar}
\label{\detokenize{thermo:tools-to-calculate-the-heat-capacity-with-pymbar}}
Shown below are functions/tools used in order to calculate
the heat capacity with pymbar.


\chapter{Utilities for the ‘foldamers’ package}
\label{\detokenize{utilities:utilities-for-the-foldamers-package}}\label{\detokenize{utilities::doc}}
This page details the functions and classes in src/util.


\section{Input/Output options (src/utilities/iotools.py)}
\label{\detokenize{utilities:input-output-options-src-utilities-iotools-py}}
Shown below is a detailed description of the input/output
options for the foldamers package.

\phantomsection\label{\detokenize{utilities:module-utilities.iotools}}\index{utilities.iotools (module)@\spxentry{utilities.iotools}\spxextra{module}}\index{write\_bonds() (in module utilities.iotools)@\spxentry{write\_bonds()}\spxextra{in module utilities.iotools}}

\begin{fulllineitems}
\phantomsection\label{\detokenize{utilities:utilities.iotools.write_bonds}}\pysiglinewithargsret{\sphinxcode{\sphinxupquote{utilities.iotools.}}\sphinxbfcode{\sphinxupquote{write\_bonds}}}{\emph{CGModel}, \emph{pdb\_object}}{}
Writes the bonds from an input CGModel class object to the file object ‘pdb\_object’, using PDB ‘CONECT’ syntax.

CGModel: Coarse grained model class object

pdb\_object: File object to which we will write the bond list

\end{fulllineitems}

\index{write\_cg\_pdb() (in module utilities.iotools)@\spxentry{write\_cg\_pdb()}\spxextra{in module utilities.iotools}}

\begin{fulllineitems}
\phantomsection\label{\detokenize{utilities:utilities.iotools.write_cg_pdb}}\pysiglinewithargsret{\sphinxcode{\sphinxupquote{utilities.iotools.}}\sphinxbfcode{\sphinxupquote{write\_cg\_pdb}}}{\emph{cgmodel}, \emph{file\_name}}{}
Writes the positions from an input CGModel class object to the file ‘filename’.  Used to test the compatibility of coarse grained model parameters with the OpenMM PDBFile() functions, which are needed to write coordinates to a PDB file during MD simulations.

CGModel: Coarse grained model class object

filename: Path to the file where we will write PDB coordinates.

\end{fulllineitems}

\index{write\_pdbfile\_without\_topology() (in module utilities.iotools)@\spxentry{write\_pdbfile\_without\_topology()}\spxextra{in module utilities.iotools}}

\begin{fulllineitems}
\phantomsection\label{\detokenize{utilities:utilities.iotools.write_pdbfile_without_topology}}\pysiglinewithargsret{\sphinxcode{\sphinxupquote{utilities.iotools.}}\sphinxbfcode{\sphinxupquote{write\_pdbfile\_without\_topology}}}{\emph{CGModel}, \emph{filename}, \emph{energy=None}}{}
Writes the positions from an input CGModel class object to the file ‘filename’.

CGModel: Coarse grained model class object

filename: Path to the file where we will write PDB coordinates.

energy: Energy to write to the PDB file, default = None

\end{fulllineitems}



\section{Utilities and random functions (src/utilities/util.py)}
\label{\detokenize{utilities:module-utilities.util}}\label{\detokenize{utilities:utilities-and-random-functions-src-utilities-util-py}}\index{utilities.util (module)@\spxentry{utilities.util}\spxextra{module}}\index{assign\_position() (in module utilities.util)@\spxentry{assign\_position()}\spxextra{in module utilities.util}}

\begin{fulllineitems}
\phantomsection\label{\detokenize{utilities:utilities.util.assign_position}}\pysiglinewithargsret{\sphinxcode{\sphinxupquote{utilities.util.}}\sphinxbfcode{\sphinxupquote{assign\_position}}}{\emph{positions}, \emph{bond\_length}, \emph{sigma}, \emph{bead\_index}, \emph{parent\_index}}{}
Assign random position for a bead

positions: Positions for all beads in the coarse-grained model.
( np.array( num\_beads x 3 ) )

bond\_length: Bond length for all beads that are bonded,
( float * simtk.unit.distance )
default = 1.0 * unit.angstrom

positions: Positions for all beads in the coarse-grained model.
( np.array( num\_beads x 3 ) )

\end{fulllineitems}

\index{assign\_position\_lattice\_style() (in module utilities.util)@\spxentry{assign\_position\_lattice\_style()}\spxextra{in module utilities.util}}

\begin{fulllineitems}
\phantomsection\label{\detokenize{utilities:utilities.util.assign_position_lattice_style}}\pysiglinewithargsret{\sphinxcode{\sphinxupquote{utilities.util.}}\sphinxbfcode{\sphinxupquote{assign\_position\_lattice\_style}}}{\emph{cgmodel}, \emph{positions}, \emph{distance\_cutoff}, \emph{bead\_index}, \emph{parent\_index}}{}
Assign random position for a bead

positions: Positions for all beads in the coarse-grained model.
( np.array( num\_beads x 3 ) )

bond\_length: Bond length for all beads that are bonded,
( float * simtk.unit.distance )
default = 1.0 * unit.angstrom

positions: Positions for all beads in the coarse-grained model.
( np.array( num\_beads x 3 ) )

\end{fulllineitems}

\index{attempt\_lattice\_move() (in module utilities.util)@\spxentry{attempt\_lattice\_move()}\spxextra{in module utilities.util}}

\begin{fulllineitems}
\phantomsection\label{\detokenize{utilities:utilities.util.attempt_lattice_move}}\pysiglinewithargsret{\sphinxcode{\sphinxupquote{utilities.util.}}\sphinxbfcode{\sphinxupquote{attempt\_lattice\_move}}}{\emph{parent\_coordinates}, \emph{bond\_length}, \emph{move\_direction\_list}}{}
Given a set of cartesian coordinates, assign a new particle
a distance of ‘bond\_length’ away in a random direction.

parent\_coordinates: Positions for a single particle,
away from which we will place a new particle a distance
of ‘bond\_length’ away.
( np.array( float * unit.angstrom ( length = 3 ) ) )

bond\_length: Bond length for all beads that are bonded,
( float * simtk.unit.distance )
default = 1.0 * unit.angstrom

trial\_coordinates: Positions for a new trial particle
( np.array( float * unit.angstrom ( length = 3 ) ) )

\end{fulllineitems}

\index{attempt\_move() (in module utilities.util)@\spxentry{attempt\_move()}\spxextra{in module utilities.util}}

\begin{fulllineitems}
\phantomsection\label{\detokenize{utilities:utilities.util.attempt_move}}\pysiglinewithargsret{\sphinxcode{\sphinxupquote{utilities.util.}}\sphinxbfcode{\sphinxupquote{attempt\_move}}}{\emph{parent\_coordinates}, \emph{bond\_length}}{}
Given a set of cartesian coordinates, assign a new particle
a distance of ‘bond\_length’ away in a random direction.

parent\_coordinates: Positions for a single particle,
away from which we will place a new particle a distance
of ‘bond\_length’ away.
( np.array( float * unit.angstrom ( length = 3 ) ) )

bond\_length: Bond length for all beads that are bonded,
( float * simtk.unit.distance )
default = 1.0 * unit.angstrom

trial\_coordinates: Positions for a new trial particle
( np.array( float * unit.angstrom ( length = 3 ) ) )

\end{fulllineitems}

\index{collisions() (in module utilities.util)@\spxentry{collisions()}\spxextra{in module utilities.util}}

\begin{fulllineitems}
\phantomsection\label{\detokenize{utilities:utilities.util.collisions}}\pysiglinewithargsret{\sphinxcode{\sphinxupquote{utilities.util.}}\sphinxbfcode{\sphinxupquote{collisions}}}{\emph{distance\_list}, \emph{distance\_cutoff}}{}
Determine whether there are any collisions between non-bonded
particles, where a “collision” is defined as a distance shorter
than the user-provided ‘bond\_length’.

distances: List of the distances between all nonbonded particles.
( list ( float * simtk.unit.distance ( length = \# nonbonded\_interactions ) ) )

bond\_length: Bond length for all beads that are bonded,
( float * simtk.unit.distance )
default = 1.0 * unit.angstrom

collision: Logical variable stating whether or not the model has
bead collisions.
default = False

\end{fulllineitems}

\index{distance() (in module utilities.util)@\spxentry{distance()}\spxextra{in module utilities.util}}

\begin{fulllineitems}
\phantomsection\label{\detokenize{utilities:utilities.util.distance}}\pysiglinewithargsret{\sphinxcode{\sphinxupquote{utilities.util.}}\sphinxbfcode{\sphinxupquote{distance}}}{\emph{positions\_1}, \emph{positions\_2}}{}
Construct a matrix of the distances between all particles.

positions\_1: Positions for a particle
( np.array( length = 3 ) )

positions\_2: Positions for a particle
( np.array( length = 3 ) )

distance
( float * unit )

\end{fulllineitems}

\index{distance\_matrix() (in module utilities.util)@\spxentry{distance\_matrix()}\spxextra{in module utilities.util}}

\begin{fulllineitems}
\phantomsection\label{\detokenize{utilities:utilities.util.distance_matrix}}\pysiglinewithargsret{\sphinxcode{\sphinxupquote{utilities.util.}}\sphinxbfcode{\sphinxupquote{distance\_matrix}}}{\emph{positions}}{}
Construct a matrix of the distances between all particles.

positions: Positions for an array of particles.
( np.array( num\_particles x 3 ) )

distance\_matrix: Matrix containing the distances between all beads.
( np.array( num\_particles x 3 ) )

\end{fulllineitems}

\index{distances() (in module utilities.util)@\spxentry{distances()}\spxextra{in module utilities.util}}

\begin{fulllineitems}
\phantomsection\label{\detokenize{utilities:utilities.util.distances}}\pysiglinewithargsret{\sphinxcode{\sphinxupquote{utilities.util.}}\sphinxbfcode{\sphinxupquote{distances}}}{\emph{interaction\_list}, \emph{positions}}{}
Calculate the distances between a trial particle (‘new\_coordinates’)
and all existing particles (‘existing\_coordinates’).

new\_coordinates: Positions for a single trial particle
( np.array( float * unit.angstrom ( length = 3 ) ) )

existing\_coordinates: Positions for a single trial particle
( np.array( float * unit.angstrom ( shape = num\_particles x 3 ) ) )

distances: List of the distances between all nonbonded particles.
( list ( float * simtk.unit.distance ( length = \# nonbonded\_interactions ) ) )

\end{fulllineitems}

\index{first\_bead() (in module utilities.util)@\spxentry{first\_bead()}\spxextra{in module utilities.util}}

\begin{fulllineitems}
\phantomsection\label{\detokenize{utilities:utilities.util.first_bead}}\pysiglinewithargsret{\sphinxcode{\sphinxupquote{utilities.util.}}\sphinxbfcode{\sphinxupquote{first\_bead}}}{\emph{positions}}{}
Determine if we have any particles in ‘positions’
Parameters
———-
positions: Positions for all beads in the coarse-grained model.
( np.array( float * unit ( shape = num\_beads x 3 ) ) )
Returns
——-
first\_bead: Logical variable stating if this is the first particle.

\end{fulllineitems}

\index{get\_move() (in module utilities.util)@\spxentry{get\_move()}\spxextra{in module utilities.util}}

\begin{fulllineitems}
\phantomsection\label{\detokenize{utilities:utilities.util.get_move}}\pysiglinewithargsret{\sphinxcode{\sphinxupquote{utilities.util.}}\sphinxbfcode{\sphinxupquote{get\_move}}}{\emph{trial\_coordinates}, \emph{move\_direction}, \emph{distance}, \emph{bond\_length}, \emph{finish\_bond=False}}{}
Given a ‘move\_direction’, a current distance, and a
target ‘bond\_length’ ( Index denoting x,y,z Cartesian 
direction), update the coordinates for the particle.

trial\_coordinates: positions for a particle
( np.array( float * unit.angstrom ( length = 3 ) ) )

move\_direction: Cartesian direction in which we will
attempt a particle placement, where: x=0, y=1, z=2. 
( integer )

distance: Current distance from parent particle
( float * simtk.unit.distance )

bond\_length: Target bond\_length for particle placement.
( float * simtk.unit.distance )

finish\_bond: Logical variable determining how we will
update the coordinates for this particle.

trial\_coordinates: Updated positions for the particle
( np.array( float * unit.angstrom ( length = 3 ) ) )

\end{fulllineitems}

\index{get\_structure\_from\_library() (in module utilities.util)@\spxentry{get\_structure\_from\_library()}\spxextra{in module utilities.util}}

\begin{fulllineitems}
\phantomsection\label{\detokenize{utilities:utilities.util.get_structure_from_library}}\pysiglinewithargsret{\sphinxcode{\sphinxupquote{utilities.util.}}\sphinxbfcode{\sphinxupquote{get\_structure\_from\_library}}}{\emph{cgmodel}}{}
Given a coarse grained model class object, this function retrieves
a set of positions for the model from the ensemble library, in:
‘../foldamers/ensembles/\$\{backbone\_length\}\_\$\{sidechain\_length\}\_\$\{sidechain\_positions\}’
If this coarse grained model does not have an ensemble library, an 
error message will be returned and we will attempt to assign 
positions at random with ‘random\_positions()’.

cgmodel: CGModel() class object.

positions: Positions for all beads in the coarse-grained model.
( np.array( num\_beads x 3 ) )

\end{fulllineitems}

\index{random\_positions() (in module utilities.util)@\spxentry{random\_positions()}\spxextra{in module utilities.util}}

\begin{fulllineitems}
\phantomsection\label{\detokenize{utilities:utilities.util.random_positions}}\pysiglinewithargsret{\sphinxcode{\sphinxupquote{utilities.util.}}\sphinxbfcode{\sphinxupquote{random\_positions}}}{\emph{cgmodel}, \emph{max\_attempts=1000}, \emph{use\_library=True}}{}
Assign random positions for all beads in a coarse-grained polymer.

cgmodel: CGModel() class object.

max\_attempts: The maximum number of times that we will attempt to build
a coarse grained model with the settings in ‘cgmodel’.
default = 1000

use\_library: A logical variable determining if we will generate a new
random structure, or take a random structure from the library in the following path:
‘../foldamers/ensembles/\$\{backbone\_length\}\_\$\{sidechain\_length\}\_\$\{sidechain\_positions\}’
default = True
( NOTE: By default, if use\_library = False, new structures will be added to the
\begin{quote}

ensemble library for the relevant coarse grained model.  If that model does not
have an ensemble library, one will be created. )
\end{quote}

positions: Positions for all beads in the coarse-grained model.
( np.array( num\_beads x 3 ) )

\end{fulllineitems}

\index{random\_sign() (in module utilities.util)@\spxentry{random\_sign()}\spxextra{in module utilities.util}}

\begin{fulllineitems}
\phantomsection\label{\detokenize{utilities:utilities.util.random_sign}}\pysiglinewithargsret{\sphinxcode{\sphinxupquote{utilities.util.}}\sphinxbfcode{\sphinxupquote{random\_sign}}}{\emph{number}}{}
Returns ‘number’ with a random sign.

number: float

number

\end{fulllineitems}



\chapter{Indices and tables}
\label{\detokenize{index:indices-and-tables}}\begin{itemize}
\item {} 
\DUrole{xref,std,std-ref}{genindex}

\item {} 
\DUrole{xref,std,std-ref}{modindex}

\item {} 
\DUrole{xref,std,std-ref}{search}

\end{itemize}


\renewcommand{\indexname}{Python Module Index}
\begin{sphinxtheindex}
\let\bigletter\sphinxstyleindexlettergroup
\bigletter{c}
\item\relax\sphinxstyleindexentry{cg\_model.cgmodel}\sphinxstyleindexpageref{cg_model:\detokenize{module-cg_model.cgmodel}}
\indexspace
\bigletter{u}
\item\relax\sphinxstyleindexentry{utilities.iotools}\sphinxstyleindexpageref{utilities:\detokenize{module-utilities.iotools}}
\item\relax\sphinxstyleindexentry{utilities.util}\sphinxstyleindexpageref{utilities:\detokenize{module-utilities.util}}
\end{sphinxtheindex}

\renewcommand{\indexname}{Index}
\printindex
\end{document}