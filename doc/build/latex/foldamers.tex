%% Generated by Sphinx.
\def\sphinxdocclass{report}
\documentclass[letterpaper,12pt,english,openany,oneside]{sphinxmanual}
\ifdefined\pdfpxdimen
   \let\sphinxpxdimen\pdfpxdimen\else\newdimen\sphinxpxdimen
\fi \sphinxpxdimen=.75bp\relax

\PassOptionsToPackage{warn}{textcomp}
\usepackage[utf8]{inputenc}
\ifdefined\DeclareUnicodeCharacter
% support both utf8 and utf8x syntaxes
  \ifdefined\DeclareUnicodeCharacterAsOptional
    \def\sphinxDUC#1{\DeclareUnicodeCharacter{"#1}}
  \else
    \let\sphinxDUC\DeclareUnicodeCharacter
  \fi
  \sphinxDUC{00A0}{\nobreakspace}
  \sphinxDUC{2500}{\sphinxunichar{2500}}
  \sphinxDUC{2502}{\sphinxunichar{2502}}
  \sphinxDUC{2514}{\sphinxunichar{2514}}
  \sphinxDUC{251C}{\sphinxunichar{251C}}
  \sphinxDUC{2572}{\textbackslash}
\fi
\usepackage{cmap}
\usepackage[T1]{fontenc}
\usepackage{amsmath,amssymb,amstext}
\usepackage{babel}



\usepackage{times}
\expandafter\ifx\csname T@LGR\endcsname\relax
\else
% LGR was declared as font encoding
  \substitutefont{LGR}{\rmdefault}{cmr}
  \substitutefont{LGR}{\sfdefault}{cmss}
  \substitutefont{LGR}{\ttdefault}{cmtt}
\fi
\expandafter\ifx\csname T@X2\endcsname\relax
  \expandafter\ifx\csname T@T2A\endcsname\relax
  \else
  % T2A was declared as font encoding
    \substitutefont{T2A}{\rmdefault}{cmr}
    \substitutefont{T2A}{\sfdefault}{cmss}
    \substitutefont{T2A}{\ttdefault}{cmtt}
  \fi
\else
% X2 was declared as font encoding
  \substitutefont{X2}{\rmdefault}{cmr}
  \substitutefont{X2}{\sfdefault}{cmss}
  \substitutefont{X2}{\ttdefault}{cmtt}
\fi


\usepackage[Bjarne]{fncychap}
\usepackage{sphinx}

\fvset{fontsize=\small}
\usepackage{geometry}

% Include hyperref last.
\usepackage{hyperref}
% Fix anchor placement for figures with captions.
\usepackage{hypcap}% it must be loaded after hyperref.
% Set up styles of URL: it should be placed after hyperref.
\urlstyle{same}

\usepackage{sphinxmessages}
\setcounter{tocdepth}{1}



\title{foldamers Documentation}
\date{Sep 05, 2019}
\release{0.0}
\author{Garrett A. Meek\\Theodore L. Fobe\\Connor M. Vogel\\Research group of Professor Michael R. Shirts\\ \\Dept. of Chemical and Biological Engineering\\University of Colorado Boulder}
\newcommand{\sphinxlogo}{\vbox{}}
\renewcommand{\releasename}{Release}
\makeindex
\begin{document}

\pagestyle{empty}
\sphinxmaketitle
\pagestyle{plain}
\sphinxtableofcontents
\pagestyle{normal}
\phantomsection\label{\detokenize{index::doc}}


This documentation is generated automatically using Sphinx, which reads all docstring-formatted comments from Python functions in the ‘foldamers’ repository.  (See foldamers/doc for Sphinx source files.)


\chapter{Coarse grained model utilities}
\label{\detokenize{cg_model:coarse-grained-model-utilities}}\label{\detokenize{cg_model::doc}}
This page details the functions and classes in src/cg\_model/cgmodel.py


\section{The ‘basic\_cgmodel’ function to build coarse grained oligomers}
\label{\detokenize{cg_model:the-basic-cgmodel-function-to-build-coarse-grained-oligomers}}
Shown below is the ‘basic\_cgmodel’ function, which requires only a minimal set of input arguments to build a coarse grained model.  Given a set of input arguments this function creates a CGModel() class object, applying a set of default values for un-defined parameters.

\phantomsection\label{\detokenize{cg_model:module-cg_model.cgmodel}}\index{cg\_model.cgmodel (module)@\spxentry{cg\_model.cgmodel}\spxextra{module}}\index{basic\_cgmodel() (in module cg\_model.cgmodel)@\spxentry{basic\_cgmodel()}\spxextra{in module cg\_model.cgmodel}}

\begin{fulllineitems}
\phantomsection\label{\detokenize{cg_model:cg_model.cgmodel.basic_cgmodel}}\pysiglinewithargsret{\sphinxcode{\sphinxupquote{cg\_model.cgmodel.}}\sphinxbfcode{\sphinxupquote{basic\_cgmodel}}}{\emph{polymer\_length=12, backbone\_length=1, sidechain\_length=1, sidechain\_positions={[}0{]}, mass=Quantity(value=100.0, unit=dalton), bond\_length=Quantity(value=0.75, unit=nanometer), sigma=Quantity(value=1.85, unit=nanometer), epsilon=Quantity(value=0.5, unit=kilocalorie/mole), positions=None}}{}~\begin{quote}\begin{description}
\item[{Parameters}] \leavevmode\begin{itemize}
\item {} 
\sphinxstyleliteralstrong{\sphinxupquote{polymer\_length}} (\sphinxstyleliteralemphasis{\sphinxupquote{integer}}) \textendash{} Number of monomer units, default = 8

\item {} 
\sphinxstyleliteralstrong{\sphinxupquote{backbone\_length}} (\sphinxhref{https://docs.python.org/3/library/functions.html\#int}{\sphinxstyleliteralemphasis{\sphinxupquote{int}}}) \textendash{} Number of beads in the backbone for individual monomers within a coarse grained model, default = 1

\item {} 
\sphinxstyleliteralstrong{\sphinxupquote{sidechain\_length}} (\sphinxhref{https://docs.python.org/3/library/functions.html\#int}{\sphinxstyleliteralemphasis{\sphinxupquote{int}}}) \textendash{} Number of beads in the sidechain for individual monomers within a coarse grained model, default = 1

\item {} 
\sphinxstyleliteralstrong{\sphinxupquote{sidechain\_positions}} (\sphinxstyleliteralemphasis{\sphinxupquote{List}}\sphinxstyleliteralemphasis{\sphinxupquote{( }}\sphinxstyleliteralemphasis{\sphinxupquote{integer}}\sphinxstyleliteralemphasis{\sphinxupquote{ )}}) \textendash{} Designates the indices of backbone beads upon which we will place sidechains, default = {[}0{]} (add a sidechain to the first backbone bead in each monomer)

\item {} 
\sphinxstyleliteralstrong{\sphinxupquote{mass}} (\sphinxhref{http://docs.openmm.org/development/api-python/generated/simtk.unit.quantity.Quantity.html}{Quantity()}) \textendash{} Mass for all coarse grained beads, default = 100.0 * unit.amu

\item {} 
\sphinxstyleliteralstrong{\sphinxupquote{bond\_length}} \textendash{} Defines the length for all bond types, default = 7.5 * unit.angstrom

\item {} 
\sphinxstyleliteralstrong{\sphinxupquote{sigma}} \textendash{} Lennard-Jones equilibrium interaction distance (by default, calculated for particles that are separated by 3 or more bonds), default = 18.5 * bond\_length (for all interaction types)

\item {} 
\sphinxstyleliteralstrong{\sphinxupquote{epsilon}} \textendash{} Lennard-Jones equilibrium interaction energy (by default, calculated for particles that are separated by 3 or more bonds), default = 0.5 * unit.kilocalorie\_per\_mole

\item {} 
\sphinxstyleliteralstrong{\sphinxupquote{positions}} \textendash{} Positions for coarse grained particles in the model, default = None

\end{itemize}

\item[{Returns}] \leavevmode
cgmodel: CGModel() class object

\end{description}\end{quote}

..warning:: this function has significant limitations, in comparison with building a coarse grained model with the CGModel() class.  In particular, this function makes it more difficult to build heteropolymers, and is best-suited for the simulation of homopolymers.
\begin{quote}\begin{description}
\item[{Example}] \leavevmode
\end{description}\end{quote}

\begin{sphinxVerbatim}[commandchars=\\\{\}]
\PYG{g+gp}{\PYGZgt{}\PYGZgt{}\PYGZgt{} }\PYG{k+kn}{from} \PYG{n+nn}{simtk} \PYG{k}{import} \PYG{n}{unit}
\PYG{g+gp}{\PYGZgt{}\PYGZgt{}\PYGZgt{} }\PYG{n}{polymer\PYGZus{}length} \PYG{o}{=} \PYG{l+m+mi}{20}
\PYG{g+gp}{\PYGZgt{}\PYGZgt{}\PYGZgt{} }\PYG{n}{backbone\PYGZus{}length} \PYG{o}{=} \PYG{l+m+mi}{1}
\PYG{g+gp}{\PYGZgt{}\PYGZgt{}\PYGZgt{} }\PYG{n}{sidechain\PYGZus{}length} \PYG{o}{=} \PYG{l+m+mi}{1}
\PYG{g+gp}{\PYGZgt{}\PYGZgt{}\PYGZgt{} }\PYG{n}{sidechain\PYGZus{}positions} \PYG{o}{=} \PYG{p}{[}\PYG{l+m+mi}{0}\PYG{p}{]}
\PYG{g+gp}{\PYGZgt{}\PYGZgt{}\PYGZgt{} }\PYG{n}{mass} \PYG{o}{=} \PYG{l+m+mf}{100.0} \PYG{o}{*} \PYG{n}{unit}\PYG{o}{.}\PYG{n}{amu}
\PYG{g+gp}{\PYGZgt{}\PYGZgt{}\PYGZgt{} }\PYG{n}{bond\PYGZus{}length}\PYG{o}{=}\PYG{l+m+mf}{0.75} \PYG{o}{*} \PYG{n}{unit}\PYG{o}{.}\PYG{n}{nanometer}
\PYG{g+gp}{\PYGZgt{}\PYGZgt{}\PYGZgt{} }\PYG{n}{sigma}\PYG{o}{=}\PYG{l+m+mf}{1.85}\PYG{o}{*}\PYG{n}{unit}\PYG{o}{.}\PYG{n}{nanometer}
\PYG{g+gp}{\PYGZgt{}\PYGZgt{}\PYGZgt{} }\PYG{n}{epsilon}\PYG{o}{=}\PYG{l+m+mf}{0.5} \PYG{o}{*} \PYG{n}{unit}\PYG{o}{.}\PYG{n}{kilocalorie\PYGZus{}per\PYGZus{}mole}
\PYG{g+gp}{\PYGZgt{}\PYGZgt{}\PYGZgt{} }\PYG{n}{cgmodel} \PYG{o}{=} \PYG{n}{basic\PYGZus{}cgmodel}\PYG{p}{(}\PYG{n}{polymer\PYGZus{}length}\PYG{o}{=}\PYG{n}{polymer\PYGZus{}length}\PYG{p}{,}\PYG{n}{backbone\PYGZus{}length}\PYG{o}{=}\PYG{n}{backbone\PYGZus{}length}\PYG{p}{,}\PYG{n}{sidechain\PYGZus{}length}\PYG{o}{=}\PYG{n}{sidechain\PYGZus{}length}\PYG{p}{,}\PYG{n}{sidechain\PYGZus{}positions}\PYG{o}{=}\PYG{n}{sidechain\PYGZus{}positions}\PYG{p}{,}\PYG{n}{mass}\PYG{o}{=}\PYG{n}{mass}\PYG{p}{,}\PYG{n}{bond\PYGZus{}length}\PYG{o}{=}\PYG{n}{bond\PYGZus{}length}\PYG{p}{,}\PYG{n}{sigma}\PYG{o}{=}\PYG{n}{sigma}\PYG{p}{,}\PYG{n}{epsilon}\PYG{o}{=}\PYG{n}{epsilon}\PYG{p}{)} 
\end{sphinxVerbatim}

\end{fulllineitems}



\section{Full ‘CGModel’ class to build/model coarse grained oligomers}
\label{\detokenize{cg_model:full-cgmodel-class-to-build-model-coarse-grained-oligomers}}
Shown below is a detailed description of the full ‘cgmodel’ class object.

\phantomsection\label{\detokenize{cg_model:module-cg_model.cgmodel}}\index{cg\_model.cgmodel (module)@\spxentry{cg\_model.cgmodel}\spxextra{module}}\index{CGModel (class in cg\_model.cgmodel)@\spxentry{CGModel}\spxextra{class in cg\_model.cgmodel}}

\begin{fulllineitems}
\phantomsection\label{\detokenize{cg_model:cg_model.cgmodel.CGModel}}\pysiglinewithargsret{\sphinxbfcode{\sphinxupquote{class }}\sphinxcode{\sphinxupquote{cg\_model.cgmodel.}}\sphinxbfcode{\sphinxupquote{CGModel}}}{\emph{positions=None, polymer\_length=12, backbone\_lengths={[}1{]}, sidechain\_lengths={[}1{]}, sidechain\_positions={[}0{]}, masses=\{'backbone\_bead\_masses': Quantity(value=100.0, unit=dalton), 'sidechain\_bead\_masses': Quantity(value=100.0, unit=dalton)\}, sigmas=\{'bb\_bb\_sigma': Quantity(value=1.875, unit=nanometer), 'bb\_sc\_sigma': Quantity(value=1.875, unit=nanometer), 'sc\_sc\_sigma': Quantity(value=1.875, unit=nanometer)\}, epsilons=\{'bb\_bb\_eps': Quantity(value=0.05, unit=kilocalorie/mole), 'sc\_sc\_eps': Quantity(value=0.05, unit=kilocalorie/mole)\}, bond\_lengths=\{'bb\_bb\_bond\_length': Quantity(value=0.75, unit=nanometer), 'bb\_sc\_bond\_length': Quantity(value=0.75, unit=nanometer), 'sc\_sc\_bond\_length': Quantity(value=0.75, unit=nanometer)\}, bond\_force\_constants=None, bond\_angle\_force\_constants=None, torsion\_force\_constants=None, equil\_bond\_angles=None, equil\_torsion\_angles=None, charges=None, constrain\_bonds=True, include\_bond\_forces=False, include\_nonbonded\_forces=True, include\_bond\_angle\_forces=True, include\_torsion\_forces=True, check\_energy\_conservation=True, use\_structure\_library=False, heteropolymer=False, monomer\_types=None, sequence=None, random\_positions=False}}{}
Build a coarse grained model class object.
\begin{quote}\begin{description}
\item[{Example}] \leavevmode
\end{description}\end{quote}

\begin{sphinxVerbatim}[commandchars=\\\{\}]
\PYG{g+gp}{\PYGZgt{}\PYGZgt{}\PYGZgt{} }\PYG{k+kn}{from} \PYG{n+nn}{foldamers}\PYG{n+nn}{.}\PYG{n+nn}{cg\PYGZus{}model}\PYG{n+nn}{.}\PYG{n+nn}{cgmodel} \PYG{k}{import} \PYG{n}{CGModel}
\PYG{g+gp}{\PYGZgt{}\PYGZgt{}\PYGZgt{} }\PYG{n}{cgmodel} \PYG{o}{=} \PYG{n}{CGModel}\PYG{p}{(}\PYG{p}{)}
\end{sphinxVerbatim}
\begin{quote}\begin{description}
\item[{Example}] \leavevmode
\end{description}\end{quote}

\begin{sphinxVerbatim}[commandchars=\\\{\}]
\PYG{g+gp}{\PYGZgt{}\PYGZgt{}\PYGZgt{} }\PYG{k+kn}{from} \PYG{n+nn}{foldamers}\PYG{n+nn}{.}\PYG{n+nn}{cg\PYGZus{}model}\PYG{n+nn}{.}\PYG{n+nn}{cgmodel} \PYG{k}{import} \PYG{n}{CGModel}
\PYG{g+gp}{\PYGZgt{}\PYGZgt{}\PYGZgt{} }\PYG{k+kn}{from} \PYG{n+nn}{simtk} \PYG{k}{import} \PYG{n}{unit}
\PYG{g+gp}{\PYGZgt{}\PYGZgt{}\PYGZgt{} }\PYG{n}{bond\PYGZus{}length} \PYG{o}{=} \PYG{l+m+mf}{7.5} \PYG{o}{*} \PYG{n}{unit}\PYG{o}{.}\PYG{n}{angstrom}
\PYG{g+gp}{\PYGZgt{}\PYGZgt{}\PYGZgt{} }\PYG{n}{bond\PYGZus{}lengths} \PYG{o}{=} \PYG{p}{\PYGZob{}}\PYG{l+s+s1}{\PYGZsq{}}\PYG{l+s+s1}{bb\PYGZus{}bb\PYGZus{}bond\PYGZus{}length}\PYG{l+s+s1}{\PYGZsq{}}\PYG{p}{:} \PYG{n}{bond\PYGZus{}length}\PYG{p}{,}\PYG{l+s+s1}{\PYGZsq{}}\PYG{l+s+s1}{bb\PYGZus{}sc\PYGZus{}bond\PYGZus{}length}\PYG{l+s+s1}{\PYGZsq{}}\PYG{p}{:} \PYG{n}{bond\PYGZus{}length}\PYG{p}{,}\PYG{l+s+s1}{\PYGZsq{}}\PYG{l+s+s1}{sc\PYGZus{}sc\PYGZus{}bond\PYGZus{}length}\PYG{l+s+s1}{\PYGZsq{}}\PYG{p}{:} \PYG{n}{bond\PYGZus{}length}\PYG{p}{\PYGZcb{}}
\PYG{g+gp}{\PYGZgt{}\PYGZgt{}\PYGZgt{} }\PYG{n}{constrain\PYGZus{}bonds} \PYG{o}{=} \PYG{k+kc}{False}
\PYG{g+gp}{\PYGZgt{}\PYGZgt{}\PYGZgt{} }\PYG{n}{cgmodel} \PYG{o}{=} \PYG{n}{CGModel}\PYG{p}{(}\PYG{n}{bond\PYGZus{}lengths}\PYG{o}{=}\PYG{n}{bond\PYGZus{}lengths}\PYG{p}{,}\PYG{n}{constrain\PYGZus{}bonds}\PYG{o}{=}\PYG{n}{constrain\PYGZus{}bonds}\PYG{p}{)}
\end{sphinxVerbatim}
\begin{quote}\begin{description}
\item[{Example}] \leavevmode
\end{description}\end{quote}

\begin{sphinxVerbatim}[commandchars=\\\{\}]
\PYG{g+gp}{\PYGZgt{}\PYGZgt{}\PYGZgt{} }\PYG{k+kn}{from} \PYG{n+nn}{foldamers}\PYG{n+nn}{.}\PYG{n+nn}{cg\PYGZus{}model}\PYG{n+nn}{.}\PYG{n+nn}{cgmodel} \PYG{k}{import} \PYG{n}{CGModel}
\PYG{g+gp}{\PYGZgt{}\PYGZgt{}\PYGZgt{} }\PYG{k+kn}{from} \PYG{n+nn}{simtk} \PYG{k}{import} \PYG{n}{unit}
\PYG{g+gp}{\PYGZgt{}\PYGZgt{}\PYGZgt{} }\PYG{n}{backbone\PYGZus{}length}\PYG{o}{=}\PYG{l+m+mi}{1}
\PYG{g+gp}{\PYGZgt{}\PYGZgt{}\PYGZgt{} }\PYG{n}{sidechain\PYGZus{}length}\PYG{o}{=}\PYG{l+m+mi}{1}
\PYG{g+gp}{\PYGZgt{}\PYGZgt{}\PYGZgt{} }\PYG{n}{sidechain\PYGZus{}positions}\PYG{o}{=}\PYG{l+m+mi}{0}
\PYG{g+gp}{\PYGZgt{}\PYGZgt{}\PYGZgt{} }\PYG{n}{bond\PYGZus{}length} \PYG{o}{=} \PYG{l+m+mf}{7.5} \PYG{o}{*} \PYG{n}{unit}\PYG{o}{.}\PYG{n}{angstrom}
\PYG{g+gp}{\PYGZgt{}\PYGZgt{}\PYGZgt{} }\PYG{n}{sigma} \PYG{o}{=} \PYG{l+m+mf}{2.0} \PYG{o}{*} \PYG{n}{bond\PYGZus{}length}
\PYG{g+gp}{\PYGZgt{}\PYGZgt{}\PYGZgt{} }\PYG{n}{epsilon} \PYG{o}{=} \PYG{l+m+mf}{0.2} \PYG{o}{*} \PYG{n}{unit}\PYG{o}{.}\PYG{n}{kilocalorie\PYGZus{}per\PYGZus{}mole}
\PYG{g+gp}{\PYGZgt{}\PYGZgt{}\PYGZgt{} }\PYG{n}{sigmas} \PYG{o}{=} \PYG{p}{\PYGZob{}}\PYG{l+s+s1}{\PYGZsq{}}\PYG{l+s+s1}{bb\PYGZus{}bb\PYGZus{}sigma}\PYG{l+s+s1}{\PYGZsq{}}\PYG{p}{:} \PYG{n}{sigma}\PYG{p}{,}\PYG{l+s+s1}{\PYGZsq{}}\PYG{l+s+s1}{sc\PYGZus{}sc\PYGZus{}sigma}\PYG{l+s+s1}{\PYGZsq{}}\PYG{p}{:} \PYG{n}{sigma}\PYG{p}{\PYGZcb{}}
\PYG{g+gp}{\PYGZgt{}\PYGZgt{}\PYGZgt{} }\PYG{n}{epsilons} \PYG{o}{=} \PYG{p}{\PYGZob{}}\PYG{l+s+s1}{\PYGZsq{}}\PYG{l+s+s1}{bb\PYGZus{}bb\PYGZus{}eps}\PYG{l+s+s1}{\PYGZsq{}}\PYG{p}{:} \PYG{n}{epsilon}\PYG{p}{,}\PYG{l+s+s1}{\PYGZsq{}}\PYG{l+s+s1}{bb\PYGZus{}sc\PYGZus{}eps}\PYG{l+s+s1}{\PYGZsq{}}\PYG{p}{:} \PYG{n}{epsilon}\PYG{p}{,}\PYG{l+s+s1}{\PYGZsq{}}\PYG{l+s+s1}{sc\PYGZus{}sc\PYGZus{}eps}\PYG{l+s+s1}{\PYGZsq{}}\PYG{p}{:} \PYG{n}{epsilon}\PYG{p}{\PYGZcb{}}
\PYG{g+gp}{\PYGZgt{}\PYGZgt{}\PYGZgt{} }\PYG{n}{A} \PYG{o}{=} \PYG{p}{\PYGZob{}}\PYG{l+s+s1}{\PYGZsq{}}\PYG{l+s+s1}{monomer\PYGZus{}name}\PYG{l+s+s1}{\PYGZsq{}}\PYG{p}{:} \PYG{l+s+s2}{\PYGZdq{}}\PYG{l+s+s2}{A}\PYG{l+s+s2}{\PYGZdq{}}\PYG{p}{,} \PYG{l+s+s1}{\PYGZsq{}}\PYG{l+s+s1}{backbone\PYGZus{}length}\PYG{l+s+s1}{\PYGZsq{}}\PYG{p}{:} \PYG{n}{backbone\PYGZus{}length}\PYG{p}{,} \PYG{l+s+s1}{\PYGZsq{}}\PYG{l+s+s1}{sidechain\PYGZus{}length}\PYG{l+s+s1}{\PYGZsq{}}\PYG{p}{:} \PYG{n}{sidechain\PYGZus{}length}\PYG{p}{,} \PYG{l+s+s1}{\PYGZsq{}}\PYG{l+s+s1}{sidechain\PYGZus{}positions}\PYG{l+s+s1}{\PYGZsq{}}\PYG{p}{:} \PYG{n}{sidechain\PYGZus{}positions}\PYG{p}{,} \PYG{l+s+s1}{\PYGZsq{}}\PYG{l+s+s1}{num\PYGZus{}beads}\PYG{l+s+s1}{\PYGZsq{}}\PYG{p}{:} \PYG{n}{num\PYGZus{}beads}\PYG{p}{,} \PYG{l+s+s1}{\PYGZsq{}}\PYG{l+s+s1}{bond\PYGZus{}lengths}\PYG{l+s+s1}{\PYGZsq{}}\PYG{p}{:} \PYG{n}{bond\PYGZus{}lengths}\PYG{p}{,} \PYG{l+s+s1}{\PYGZsq{}}\PYG{l+s+s1}{epsilons}\PYG{l+s+s1}{\PYGZsq{}}\PYG{p}{:} \PYG{n}{epsilons}\PYG{p}{,} \PYG{l+s+s1}{\PYGZsq{}}\PYG{l+s+s1}{sigmas}\PYG{l+s+s1}{\PYGZsq{}}\PYG{p}{:} \PYG{n}{sigmas}\PYG{p}{\PYGZcb{}}
\PYG{g+gp}{\PYGZgt{}\PYGZgt{}\PYGZgt{} }\PYG{n}{B} \PYG{o}{=} \PYG{p}{\PYGZob{}}\PYG{l+s+s1}{\PYGZsq{}}\PYG{l+s+s1}{monomer\PYGZus{}name}\PYG{l+s+s1}{\PYGZsq{}}\PYG{p}{:} \PYG{l+s+s2}{\PYGZdq{}}\PYG{l+s+s2}{B}\PYG{l+s+s2}{\PYGZdq{}}\PYG{p}{,} \PYG{l+s+s1}{\PYGZsq{}}\PYG{l+s+s1}{backbone\PYGZus{}length}\PYG{l+s+s1}{\PYGZsq{}}\PYG{p}{:} \PYG{n}{backbone\PYGZus{}length}\PYG{p}{,} \PYG{l+s+s1}{\PYGZsq{}}\PYG{l+s+s1}{sidechain\PYGZus{}length}\PYG{l+s+s1}{\PYGZsq{}}\PYG{p}{:} \PYG{n}{sidechain\PYGZus{}length}\PYG{p}{,} \PYG{l+s+s1}{\PYGZsq{}}\PYG{l+s+s1}{sidechain\PYGZus{}positions}\PYG{l+s+s1}{\PYGZsq{}}\PYG{p}{:} \PYG{n}{sidechain\PYGZus{}positions}\PYG{p}{,} \PYG{l+s+s1}{\PYGZsq{}}\PYG{l+s+s1}{num\PYGZus{}beads}\PYG{l+s+s1}{\PYGZsq{}}\PYG{p}{:} \PYG{n}{num\PYGZus{}beads}\PYG{p}{,} \PYG{l+s+s1}{\PYGZsq{}}\PYG{l+s+s1}{bond\PYGZus{}lengths}\PYG{l+s+s1}{\PYGZsq{}}\PYG{p}{:} \PYG{n}{bond\PYGZus{}lengths}\PYG{p}{,} \PYG{l+s+s1}{\PYGZsq{}}\PYG{l+s+s1}{epsilons}\PYG{l+s+s1}{\PYGZsq{}}\PYG{p}{:} \PYG{n}{epsilons}\PYG{p}{,} \PYG{l+s+s1}{\PYGZsq{}}\PYG{l+s+s1}{sigmas}\PYG{l+s+s1}{\PYGZsq{}}\PYG{p}{:} \PYG{n}{sigmas}\PYG{p}{\PYGZcb{}}
\PYG{g+gp}{\PYGZgt{}\PYGZgt{}\PYGZgt{} }\PYG{n}{monomer\PYGZus{}types} \PYG{o}{=} \PYG{p}{[}\PYG{n}{A}\PYG{p}{,}\PYG{n}{B}\PYG{p}{]}
\PYG{g+gp}{\PYGZgt{}\PYGZgt{}\PYGZgt{} }\PYG{n}{sequence} \PYG{o}{=} \PYG{p}{[}\PYG{n}{A}\PYG{p}{,}\PYG{n}{A}\PYG{p}{,}\PYG{n}{A}\PYG{p}{,}\PYG{n}{B}\PYG{p}{,}\PYG{n}{A}\PYG{p}{,}\PYG{n}{A}\PYG{p}{,}\PYG{n}{A}\PYG{p}{,}\PYG{n}{B}\PYG{p}{,}\PYG{n}{A}\PYG{p}{,}\PYG{n}{A}\PYG{p}{,}\PYG{n}{A}\PYG{p}{,}\PYG{n}{B}\PYG{p}{]}
\PYG{g+gp}{\PYGZgt{}\PYGZgt{}\PYGZgt{} }\PYG{n}{cgmodel} \PYG{o}{=} \PYG{n}{CGModel}\PYG{p}{(}\PYG{n}{heteropolymer}\PYG{o}{=}\PYG{k+kc}{True}\PYG{p}{,}\PYG{n}{monomer\PYGZus{}types}\PYG{o}{=}\PYG{n}{monomer\PYGZus{}types}\PYG{p}{,}\PYG{n}{sequence}\PYG{o}{=}\PYG{n}{sequence}\PYG{p}{)}
\end{sphinxVerbatim}
\index{get\_all\_particle\_masses() (cg\_model.cgmodel.CGModel method)@\spxentry{get\_all\_particle\_masses()}\spxextra{cg\_model.cgmodel.CGModel method}}

\begin{fulllineitems}
\phantomsection\label{\detokenize{cg_model:cg_model.cgmodel.CGModel.get_all_particle_masses}}\pysiglinewithargsret{\sphinxbfcode{\sphinxupquote{get\_all\_particle\_masses}}}{}{}
Returns a list of all unique particle masses
\begin{quote}\begin{description}
\item[{Parameters}] \leavevmode
\sphinxstyleliteralstrong{\sphinxupquote{CGModel}} (\sphinxstyleliteralemphasis{\sphinxupquote{class}}) \textendash{} CGModel() class object

\item[{Returns}] \leavevmode
list\_of\_masses: List of unique particle masses

\item[{Return type}] \leavevmode

List( \sphinxhref{https://docs.openmm.org/development/api-python/generated/simtk.unit.quantity.Quantity.html}{Quantity()} )


\end{description}\end{quote}

\end{fulllineitems}

\index{get\_bond\_angle\_force\_constant() (cg\_model.cgmodel.CGModel method)@\spxentry{get\_bond\_angle\_force\_constant()}\spxextra{cg\_model.cgmodel.CGModel method}}

\begin{fulllineitems}
\phantomsection\label{\detokenize{cg_model:cg_model.cgmodel.CGModel.get_bond_angle_force_constant}}\pysiglinewithargsret{\sphinxbfcode{\sphinxupquote{get\_bond\_angle\_force\_constant}}}{\emph{particle\_1\_index}, \emph{particle\_2\_index}, \emph{particle\_3\_index}}{}
Determines the correct bond angle force constant for a bond angle between three particles, given their indices within the coarse grained model
\begin{quote}\begin{description}
\item[{Parameters}] \leavevmode\begin{itemize}
\item {} 
\sphinxstyleliteralstrong{\sphinxupquote{CGModel}} (\sphinxstyleliteralemphasis{\sphinxupquote{class}}) \textendash{} CGModel() class object

\item {} 
\sphinxstyleliteralstrong{\sphinxupquote{particle\_1\_index}} (\sphinxhref{https://docs.python.org/3/library/functions.html\#int}{\sphinxstyleliteralemphasis{\sphinxupquote{int}}}) \textendash{} Index for the first particle

\item {} 
\sphinxstyleliteralstrong{\sphinxupquote{particle\_2\_index}} (\sphinxhref{https://docs.python.org/3/library/functions.html\#int}{\sphinxstyleliteralemphasis{\sphinxupquote{int}}}) \textendash{} Index for the second particle

\item {} 
\sphinxstyleliteralstrong{\sphinxupquote{particle\_3\_index}} (\sphinxhref{https://docs.python.org/3/library/functions.html\#int}{\sphinxstyleliteralemphasis{\sphinxupquote{int}}}) \textendash{} Index for the third particle

\end{itemize}

\item[{Returns}] \leavevmode
bond\_angle\_force\_constant: The assigned bond angle force constant for the provided particles

\item[{Return type}] \leavevmode

bond\_angle\_force\_constant: \sphinxhref{https://docs.openmm.org/development/api-python/generated/simtk.unit.quantity.Quantity.html}{Quantity()}


\end{description}\end{quote}

\end{fulllineitems}

\index{get\_bond\_angle\_list() (cg\_model.cgmodel.CGModel method)@\spxentry{get\_bond\_angle\_list()}\spxextra{cg\_model.cgmodel.CGModel method}}

\begin{fulllineitems}
\phantomsection\label{\detokenize{cg_model:cg_model.cgmodel.CGModel.get_bond_angle_list}}\pysiglinewithargsret{\sphinxbfcode{\sphinxupquote{get\_bond\_angle\_list}}}{}{}
Construct a list of bond angles, which can be used to build bond angle potentials for the coarse grained model
\begin{quote}\begin{description}
\item[{Parameters}] \leavevmode
\sphinxstyleliteralstrong{\sphinxupquote{CGModel}} (\sphinxstyleliteralemphasis{\sphinxupquote{class}}) \textendash{} CGModel() class object

\item[{Returns}] \leavevmode
A list of indices for all of the bond angles in the coarse grained model

\item[{Return type}] \leavevmode
List( List( \sphinxhref{https://docs.python.org/3/library/functions.html\#int}{int}, \sphinxhref{https://docs.python.org/3/library/functions.html\#int}{int}, \sphinxhref{https://docs.python.org/3/library/functions.html\#int}{int} ) )

\end{description}\end{quote}

\end{fulllineitems}

\index{get\_bond\_force\_constant() (cg\_model.cgmodel.CGModel method)@\spxentry{get\_bond\_force\_constant()}\spxextra{cg\_model.cgmodel.CGModel method}}

\begin{fulllineitems}
\phantomsection\label{\detokenize{cg_model:cg_model.cgmodel.CGModel.get_bond_force_constant}}\pysiglinewithargsret{\sphinxbfcode{\sphinxupquote{get\_bond\_force\_constant}}}{\emph{particle\_1\_index}, \emph{particle\_2\_index}}{}
Determines the correct bond force constant for two particles, given their indices
\begin{quote}\begin{description}
\item[{Parameters}] \leavevmode\begin{itemize}
\item {} 
\sphinxstyleliteralstrong{\sphinxupquote{CGModel}} (\sphinxstyleliteralemphasis{\sphinxupquote{class}}) \textendash{} CGModel() class object

\item {} 
\sphinxstyleliteralstrong{\sphinxupquote{particle\_1\_index}} (\sphinxhref{https://docs.python.org/3/library/functions.html\#int}{\sphinxstyleliteralemphasis{\sphinxupquote{int}}}) \textendash{} Index for the first particle

\item {} 
\sphinxstyleliteralstrong{\sphinxupquote{particle\_2\_index}} (\sphinxhref{https://docs.python.org/3/library/functions.html\#int}{\sphinxstyleliteralemphasis{\sphinxupquote{int}}}) \textendash{} Index for the second particle

\end{itemize}

\item[{Returns}] \leavevmode
bond\_force\_constant: The assigned bond force constant for the provided particles

\item[{Return type}] \leavevmode

bond\_length: \sphinxhref{https://docs.openmm.org/development/api-python/generated/simtk.unit.quantity.Quantity.html}{Quantity()}


\end{description}\end{quote}

\end{fulllineitems}

\index{get\_bond\_length() (cg\_model.cgmodel.CGModel method)@\spxentry{get\_bond\_length()}\spxextra{cg\_model.cgmodel.CGModel method}}

\begin{fulllineitems}
\phantomsection\label{\detokenize{cg_model:cg_model.cgmodel.CGModel.get_bond_length}}\pysiglinewithargsret{\sphinxbfcode{\sphinxupquote{get\_bond\_length}}}{\emph{particle\_1\_index}, \emph{particle\_2\_index}}{}
Determines the correct bond length for two particles, given their indices.
\begin{quote}\begin{description}
\item[{Parameters}] \leavevmode\begin{itemize}
\item {} 
\sphinxstyleliteralstrong{\sphinxupquote{CGModel}} (\sphinxstyleliteralemphasis{\sphinxupquote{class}}) \textendash{} CGModel() class object

\item {} 
\sphinxstyleliteralstrong{\sphinxupquote{particle\_1\_index}} (\sphinxhref{https://docs.python.org/3/library/functions.html\#int}{\sphinxstyleliteralemphasis{\sphinxupquote{int}}}) \textendash{} Index for the first particle

\item {} 
\sphinxstyleliteralstrong{\sphinxupquote{particle\_2\_index}} (\sphinxhref{https://docs.python.org/3/library/functions.html\#int}{\sphinxstyleliteralemphasis{\sphinxupquote{int}}}) \textendash{} Index for the second particle

\end{itemize}

\item[{Returns}] \leavevmode
bond\_length: The assigned bond length for the provided particles

\item[{Return type}] \leavevmode

bond\_length: \sphinxhref{https://docs.openmm.org/development/api-python/generated/simtk.unit.quantity.Quantity.html}{Quantity()}


\end{description}\end{quote}

\end{fulllineitems}

\index{get\_bond\_length\_from\_names() (cg\_model.cgmodel.CGModel method)@\spxentry{get\_bond\_length\_from\_names()}\spxextra{cg\_model.cgmodel.CGModel method}}

\begin{fulllineitems}
\phantomsection\label{\detokenize{cg_model:cg_model.cgmodel.CGModel.get_bond_length_from_names}}\pysiglinewithargsret{\sphinxbfcode{\sphinxupquote{get\_bond\_length\_from\_names}}}{\emph{particle\_1\_name}, \emph{particle\_2\_name}}{}
Determines the correct bond length for two particles, given their symbols.
\begin{quote}\begin{description}
\item[{Parameters}] \leavevmode\begin{itemize}
\item {} 
\sphinxstyleliteralstrong{\sphinxupquote{CGModel}} (\sphinxstyleliteralemphasis{\sphinxupquote{class}}) \textendash{} CGModel() class object

\item {} 
\sphinxstyleliteralstrong{\sphinxupquote{particle\_1\_name}} (\sphinxhref{https://docs.python.org/3/library/stdtypes.html\#str}{\sphinxstyleliteralemphasis{\sphinxupquote{str}}}) \textendash{} Name for the first particle

\item {} 
\sphinxstyleliteralstrong{\sphinxupquote{particle\_2\_name}} (\sphinxhref{https://docs.python.org/3/library/stdtypes.html\#str}{\sphinxstyleliteralemphasis{\sphinxupquote{str}}}) \textendash{} Name for the second particle

\end{itemize}

\item[{Returns}] \leavevmode
bond\_length: The assigned bond length for the provided particles

\item[{Return type}] \leavevmode

bond\_length: \sphinxhref{https://docs.openmm.org/development/api-python/generated/simtk.unit.quantity.Quantity.html}{Quantity()}


\end{description}\end{quote}

\end{fulllineitems}

\index{get\_bond\_list() (cg\_model.cgmodel.CGModel method)@\spxentry{get\_bond\_list()}\spxextra{cg\_model.cgmodel.CGModel method}}

\begin{fulllineitems}
\phantomsection\label{\detokenize{cg_model:cg_model.cgmodel.CGModel.get_bond_list}}\pysiglinewithargsret{\sphinxbfcode{\sphinxupquote{get\_bond\_list}}}{}{}
Construct a bond list for the coarse grained model
\begin{quote}\begin{description}
\item[{Parameters}] \leavevmode
\sphinxstyleliteralstrong{\sphinxupquote{CGModel}} (\sphinxstyleliteralemphasis{\sphinxupquote{class}}) \textendash{} CGModel() class object

\item[{Returns}] \leavevmode
bond\_list: A list of the bonds in the coarse grained model.

\item[{Return type}] \leavevmode
bond\_list: List( List( \sphinxhref{https://docs.python.org/3/library/functions.html\#int}{int}, \sphinxhref{https://docs.python.org/3/library/functions.html\#int}{int} ) )

\end{description}\end{quote}

\end{fulllineitems}

\index{get\_epsilon() (cg\_model.cgmodel.CGModel method)@\spxentry{get\_epsilon()}\spxextra{cg\_model.cgmodel.CGModel method}}

\begin{fulllineitems}
\phantomsection\label{\detokenize{cg_model:cg_model.cgmodel.CGModel.get_epsilon}}\pysiglinewithargsret{\sphinxbfcode{\sphinxupquote{get\_epsilon}}}{\emph{particle\_index}, \emph{particle\_type=None}}{}
Returns the Lennard-Jones potential epsilon value for a particle, given its index within the coarse grained model.
\begin{quote}\begin{description}
\item[{Parameters}] \leavevmode\begin{itemize}
\item {} 
\sphinxstyleliteralstrong{\sphinxupquote{CGModel}} (\sphinxstyleliteralemphasis{\sphinxupquote{class}}) \textendash{} CGModel() class object

\item {} 
\sphinxstyleliteralstrong{\sphinxupquote{particle\_index}} (\sphinxhref{https://docs.python.org/3/library/functions.html\#int}{\sphinxstyleliteralemphasis{\sphinxupquote{int}}}) \textendash{} Index of the particle for which we would like to determine the type

\item {} 
\sphinxstyleliteralstrong{\sphinxupquote{particle\_type}} (\sphinxhref{https://docs.python.org/3/library/stdtypes.html\#str}{\sphinxstyleliteralemphasis{\sphinxupquote{str}}}) \textendash{} Designates a particle as “backbone” or “sidechain”

\end{itemize}

\item[{Returns}] \leavevmode
epsilon: The assigned Lennard-Jones epsilon value for the provided particle index

\item[{Return type}] \leavevmode

\sphinxhref{https://docs.openmm.org/development/api-python/generated/simtk.unit.quantity.Quantity.html}{Quantity()}


\end{description}\end{quote}

\end{fulllineitems}

\index{get\_equil\_bond\_angle() (cg\_model.cgmodel.CGModel method)@\spxentry{get\_equil\_bond\_angle()}\spxextra{cg\_model.cgmodel.CGModel method}}

\begin{fulllineitems}
\phantomsection\label{\detokenize{cg_model:cg_model.cgmodel.CGModel.get_equil_bond_angle}}\pysiglinewithargsret{\sphinxbfcode{\sphinxupquote{get\_equil\_bond\_angle}}}{\emph{particle\_1\_index}, \emph{particle\_2\_index}, \emph{particle\_3\_index}}{}
Determines the correct equilibrium bond angle between three particles, given their indices within the coarse grained model
\begin{quote}\begin{description}
\item[{Parameters}] \leavevmode\begin{itemize}
\item {} 
\sphinxstyleliteralstrong{\sphinxupquote{CGModel}} (\sphinxstyleliteralemphasis{\sphinxupquote{class}}) \textendash{} CGModel() class object

\item {} 
\sphinxstyleliteralstrong{\sphinxupquote{particle\_1\_index}} (\sphinxhref{https://docs.python.org/3/library/functions.html\#int}{\sphinxstyleliteralemphasis{\sphinxupquote{int}}}) \textendash{} Index for the first particle

\item {} 
\sphinxstyleliteralstrong{\sphinxupquote{particle\_2\_index}} (\sphinxhref{https://docs.python.org/3/library/functions.html\#int}{\sphinxstyleliteralemphasis{\sphinxupquote{int}}}) \textendash{} Index for the second particle

\item {} 
\sphinxstyleliteralstrong{\sphinxupquote{particle\_3\_index}} (\sphinxhref{https://docs.python.org/3/library/functions.html\#int}{\sphinxstyleliteralemphasis{\sphinxupquote{int}}}) \textendash{} Index for the third particle

\end{itemize}

\item[{Returns}] \leavevmode
equil\_bond\_angle: The assigned equilibrium bond angle for the provided particles

\item[{Return type}] \leavevmode
equil\_bond\_angle: float

\end{description}\end{quote}

\end{fulllineitems}

\index{get\_equil\_torsion\_angle() (cg\_model.cgmodel.CGModel method)@\spxentry{get\_equil\_torsion\_angle()}\spxextra{cg\_model.cgmodel.CGModel method}}

\begin{fulllineitems}
\phantomsection\label{\detokenize{cg_model:cg_model.cgmodel.CGModel.get_equil_torsion_angle}}\pysiglinewithargsret{\sphinxbfcode{\sphinxupquote{get\_equil\_torsion\_angle}}}{\emph{torsion}}{}
Determines the correct equilibrium angle for a torsion (bond angle involving four particles), given their indices within the coarse grained model
\begin{quote}\begin{description}
\item[{Parameters}] \leavevmode\begin{itemize}
\item {} 
\sphinxstyleliteralstrong{\sphinxupquote{CGModel}} (\sphinxstyleliteralemphasis{\sphinxupquote{class}}) \textendash{} CGModel() class object

\item {} 
\sphinxstyleliteralstrong{\sphinxupquote{torsion}} (\sphinxstyleliteralemphasis{\sphinxupquote{List}}\sphinxstyleliteralemphasis{\sphinxupquote{( }}\sphinxhref{https://docs.python.org/3/library/functions.html\#int}{\sphinxstyleliteralemphasis{\sphinxupquote{int}}}\sphinxstyleliteralemphasis{\sphinxupquote{ )}}) \textendash{} A list of the indices for the particles in a torsion

\end{itemize}

\item[{Returns}] \leavevmode
equil\_torsion\_angle: The assigned equilibrium torsion angle for the provided particles

\item[{Return type}] \leavevmode
equil\_torsion\_angle: float

\end{description}\end{quote}

\end{fulllineitems}

\index{get\_monomer\_types() (cg\_model.cgmodel.CGModel method)@\spxentry{get\_monomer\_types()}\spxextra{cg\_model.cgmodel.CGModel method}}

\begin{fulllineitems}
\phantomsection\label{\detokenize{cg_model:cg_model.cgmodel.CGModel.get_monomer_types}}\pysiglinewithargsret{\sphinxbfcode{\sphinxupquote{get\_monomer\_types}}}{}{}
Get a list of ‘monomer\_types’ for all unique monomers.
\begin{quote}\begin{description}
\item[{Parameters}] \leavevmode
\sphinxstyleliteralstrong{\sphinxupquote{CGModel}} (\sphinxstyleliteralemphasis{\sphinxupquote{class}}) \textendash{} CGModel() class object

\item[{Returns}] \leavevmode
monomer\_types: A list of unique monomer types in the coarse grained model

\item[{Return type}] \leavevmode

monomer\_types: List( dict( ‘monomer\_name’: str, ‘backbone\_length’: int, ‘sidechain\_length’: int, ‘sidechain\_positions’: List( int ), ‘num\_beads’: int, ‘bond\_lengths’: List( \sphinxhref{https://docs.openmm.org/development/api-python/generated/simtk.unit.quantity.Quantity.html}{Quantity()} ), ‘epsilons’: List( \sphinxhref{https://docs.openmm.org/development/api-python/generated/simtk.unit.quantity.Quantity.html}{Quantity()} ), ‘sigmas’: List( \sphinxhref{https://docs.openmm.org/development/api-python/generated/simtk.unit.quantity.Quantity.html}{Quantity()} ) ) )


\end{description}\end{quote}

\end{fulllineitems}

\index{get\_nonbonded\_exclusion\_list() (cg\_model.cgmodel.CGModel method)@\spxentry{get\_nonbonded\_exclusion\_list()}\spxextra{cg\_model.cgmodel.CGModel method}}

\begin{fulllineitems}
\phantomsection\label{\detokenize{cg_model:cg_model.cgmodel.CGModel.get_nonbonded_exclusion_list}}\pysiglinewithargsret{\sphinxbfcode{\sphinxupquote{get\_nonbonded\_exclusion\_list}}}{}{}
Get a list of the nonbonded interaction exclusions, which are assigned if two particles are separated by less than three bonds
\begin{quote}\begin{description}
\item[{Parameters}] \leavevmode
\sphinxstyleliteralstrong{\sphinxupquote{CGModel}} (\sphinxstyleliteralemphasis{\sphinxupquote{class}}) \textendash{} CGModel() class object

\item[{Returns}] \leavevmode
exclusion\_list: A list of the nonbonded particle interaction exclusions for the coarse grained model

\item[{Return type}] \leavevmode
List( List( \sphinxhref{https://docs.python.org/3/library/functions.html\#int}{int}, \sphinxhref{https://docs.python.org/3/library/functions.html\#int}{int} ) )

\end{description}\end{quote}

\end{fulllineitems}

\index{get\_nonbonded\_interaction\_list() (cg\_model.cgmodel.CGModel method)@\spxentry{get\_nonbonded\_interaction\_list()}\spxextra{cg\_model.cgmodel.CGModel method}}

\begin{fulllineitems}
\phantomsection\label{\detokenize{cg_model:cg_model.cgmodel.CGModel.get_nonbonded_interaction_list}}\pysiglinewithargsret{\sphinxbfcode{\sphinxupquote{get\_nonbonded\_interaction\_list}}}{}{}
Construct a nonbonded interaction list for the coarse grained model
\begin{quote}\begin{description}
\item[{Parameters}] \leavevmode
\sphinxstyleliteralstrong{\sphinxupquote{CGModel}} (\sphinxstyleliteralemphasis{\sphinxupquote{class}}) \textendash{} CGModel() class object

\item[{Returns}] \leavevmode
interaction\_list: A list of the nonbonded interactions (which don’t violate exclusion rules) in the coarse grained model

\item[{Return type}] \leavevmode
interaction\_list: List( List( \sphinxhref{https://docs.python.org/3/library/functions.html\#int}{int}, \sphinxhref{https://docs.python.org/3/library/functions.html\#int}{int} ) )

\end{description}\end{quote}

\end{fulllineitems}

\index{get\_num\_beads() (cg\_model.cgmodel.CGModel method)@\spxentry{get\_num\_beads()}\spxextra{cg\_model.cgmodel.CGModel method}}

\begin{fulllineitems}
\phantomsection\label{\detokenize{cg_model:cg_model.cgmodel.CGModel.get_num_beads}}\pysiglinewithargsret{\sphinxbfcode{\sphinxupquote{get\_num\_beads}}}{}{}
Calculate the number of beads in a coarse grained model class object
\begin{quote}\begin{description}
\item[{Parameters}] \leavevmode
\sphinxstyleliteralstrong{\sphinxupquote{CGModel}} (\sphinxstyleliteralemphasis{\sphinxupquote{class}}) \textendash{} CGModel() class object

\item[{Returns}] \leavevmode
num\_beads: The total number of beads in the coarse grained model

\item[{Return type}] \leavevmode
num\_beads: int

\end{description}\end{quote}

\end{fulllineitems}

\index{get\_particle\_charge() (cg\_model.cgmodel.CGModel method)@\spxentry{get\_particle\_charge()}\spxextra{cg\_model.cgmodel.CGModel method}}

\begin{fulllineitems}
\phantomsection\label{\detokenize{cg_model:cg_model.cgmodel.CGModel.get_particle_charge}}\pysiglinewithargsret{\sphinxbfcode{\sphinxupquote{get\_particle\_charge}}}{\emph{particle\_index}}{}
Returns the charge for a particle, given its index within the coarse grained model
\begin{quote}\begin{description}
\item[{Parameters}] \leavevmode\begin{itemize}
\item {} 
\sphinxstyleliteralstrong{\sphinxupquote{CGModel}} (\sphinxstyleliteralemphasis{\sphinxupquote{class}}) \textendash{} CGModel() class object

\item {} 
\sphinxstyleliteralstrong{\sphinxupquote{particle\_index}} (\sphinxhref{https://docs.python.org/3/library/functions.html\#int}{\sphinxstyleliteralemphasis{\sphinxupquote{int}}}) \textendash{} Index of the particle for which we would like to determine the type

\end{itemize}

\item[{Returns}] \leavevmode
particle\_charge: The charge for the provided particle index

\item[{Return type}] \leavevmode

\sphinxhref{https://docs.openmm.org/development/api-python/generated/simtk.unit.quantity.Quantity.html}{Quantity()}


\end{description}\end{quote}

\end{fulllineitems}

\index{get\_particle\_list() (cg\_model.cgmodel.CGModel method)@\spxentry{get\_particle\_list()}\spxextra{cg\_model.cgmodel.CGModel method}}

\begin{fulllineitems}
\phantomsection\label{\detokenize{cg_model:cg_model.cgmodel.CGModel.get_particle_list}}\pysiglinewithargsret{\sphinxbfcode{\sphinxupquote{get\_particle\_list}}}{}{}
Get a list of particles, where the indices correspond to those in the system/topology.
\begin{quote}\begin{description}
\item[{Parameters}] \leavevmode
\sphinxstyleliteralstrong{\sphinxupquote{CGModel}} (\sphinxstyleliteralemphasis{\sphinxupquote{class}}) \textendash{} CGModel() class object

\item[{Returns}] \leavevmode
particle\_list: A list of unique particles in the coarse grained model

\item[{Return type}] \leavevmode
particle\_list: List( \sphinxhref{https://docs.python.org/3/library/stdtypes.html\#str}{str} )

\end{description}\end{quote}

\end{fulllineitems}

\index{get\_particle\_mass() (cg\_model.cgmodel.CGModel method)@\spxentry{get\_particle\_mass()}\spxextra{cg\_model.cgmodel.CGModel method}}

\begin{fulllineitems}
\phantomsection\label{\detokenize{cg_model:cg_model.cgmodel.CGModel.get_particle_mass}}\pysiglinewithargsret{\sphinxbfcode{\sphinxupquote{get\_particle\_mass}}}{\emph{particle\_index}}{}
Get the mass for a particle, given its index within the coarse grained model
\begin{quote}\begin{description}
\item[{Parameters}] \leavevmode\begin{itemize}
\item {} 
\sphinxstyleliteralstrong{\sphinxupquote{CGModel}} (\sphinxstyleliteralemphasis{\sphinxupquote{class}}) \textendash{} CGModel() class object

\item {} 
\sphinxstyleliteralstrong{\sphinxupquote{particle\_index}} (\sphinxhref{https://docs.python.org/3/library/functions.html\#int}{\sphinxstyleliteralemphasis{\sphinxupquote{int}}}) \textendash{} Index of the particle for which we would like to determine the type

\end{itemize}

\item[{Returns}] \leavevmode
particle\_mass: The mass for the provided particle index

\item[{Return type}] \leavevmode

\sphinxhref{https://docs.openmm.org/development/api-python/generated/simtk.unit.quantity.Quantity.html}{Quantity()}


\end{description}\end{quote}

\end{fulllineitems}

\index{get\_particle\_name() (cg\_model.cgmodel.CGModel method)@\spxentry{get\_particle\_name()}\spxextra{cg\_model.cgmodel.CGModel method}}

\begin{fulllineitems}
\phantomsection\label{\detokenize{cg_model:cg_model.cgmodel.CGModel.get_particle_name}}\pysiglinewithargsret{\sphinxbfcode{\sphinxupquote{get\_particle\_name}}}{\emph{particle\_index}}{}
Returns the name of a particle, given its index within the model
\begin{quote}\begin{description}
\item[{Parameters}] \leavevmode\begin{itemize}
\item {} 
\sphinxstyleliteralstrong{\sphinxupquote{CGModel}} (\sphinxstyleliteralemphasis{\sphinxupquote{class}}) \textendash{} CGModel() class object

\item {} 
\sphinxstyleliteralstrong{\sphinxupquote{particle\_index}} (\sphinxhref{https://docs.python.org/3/library/functions.html\#int}{\sphinxstyleliteralemphasis{\sphinxupquote{int}}}) \textendash{} Index of the particle for which we would like to determine the type

\end{itemize}

\item[{Returns}] \leavevmode
particle\_name: The name of the particle

\item[{Return type}] \leavevmode
particle\_name: str

\end{description}\end{quote}

\end{fulllineitems}

\index{get\_particle\_type() (cg\_model.cgmodel.CGModel method)@\spxentry{get\_particle\_type()}\spxextra{cg\_model.cgmodel.CGModel method}}

\begin{fulllineitems}
\phantomsection\label{\detokenize{cg_model:cg_model.cgmodel.CGModel.get_particle_type}}\pysiglinewithargsret{\sphinxbfcode{\sphinxupquote{get\_particle\_type}}}{\emph{particle\_index}, \emph{particle\_name=None}}{}
Indicates if a particle is a backbone bead or a sidechain bead
\begin{quote}\begin{description}
\item[{Parameters}] \leavevmode\begin{itemize}
\item {} 
\sphinxstyleliteralstrong{\sphinxupquote{CGModel}} (\sphinxstyleliteralemphasis{\sphinxupquote{class}}) \textendash{} CGModel() class object

\item {} 
\sphinxstyleliteralstrong{\sphinxupquote{particle\_index}} (\sphinxhref{https://docs.python.org/3/library/functions.html\#int}{\sphinxstyleliteralemphasis{\sphinxupquote{int}}}) \textendash{} Index of the particle for which we would like to determine the type

\item {} 
\sphinxstyleliteralstrong{\sphinxupquote{particle\_name}} (\sphinxhref{https://docs.python.org/3/library/stdtypes.html\#str}{\sphinxstyleliteralemphasis{\sphinxupquote{str}}}) \textendash{} Name of the particle that we would like to “type”.

\end{itemize}

\item[{Returns}] \leavevmode
particle\_type: ‘backbone’ or ‘sidechain’

\item[{Return type}] \leavevmode
particle\_type: str

\end{description}\end{quote}

\end{fulllineitems}

\index{get\_sigma() (cg\_model.cgmodel.CGModel method)@\spxentry{get\_sigma()}\spxextra{cg\_model.cgmodel.CGModel method}}

\begin{fulllineitems}
\phantomsection\label{\detokenize{cg_model:cg_model.cgmodel.CGModel.get_sigma}}\pysiglinewithargsret{\sphinxbfcode{\sphinxupquote{get\_sigma}}}{\emph{particle\_index}, \emph{particle\_type=None}}{}
Returns the Lennard-Jones potential sigma value for a particle, given its index within the coarse grained model.
\begin{quote}\begin{description}
\item[{Parameters}] \leavevmode\begin{itemize}
\item {} 
\sphinxstyleliteralstrong{\sphinxupquote{CGModel}} (\sphinxstyleliteralemphasis{\sphinxupquote{class}}) \textendash{} CGModel() class object

\item {} 
\sphinxstyleliteralstrong{\sphinxupquote{particle\_index}} (\sphinxhref{https://docs.python.org/3/library/functions.html\#int}{\sphinxstyleliteralemphasis{\sphinxupquote{int}}}) \textendash{} Index of the particle for which we would like to determine the type

\item {} 
\sphinxstyleliteralstrong{\sphinxupquote{particle\_type}} (\sphinxhref{https://docs.python.org/3/library/stdtypes.html\#str}{\sphinxstyleliteralemphasis{\sphinxupquote{str}}}) \textendash{} Designates a particle as “backbone” or “sidechain”

\end{itemize}

\item[{Returns}] \leavevmode
sigma: The assigned Lennard-Jones sigma value for the provided particle index

\item[{Return type}] \leavevmode

\sphinxhref{https://docs.openmm.org/development/api-python/generated/simtk.unit.quantity.Quantity.html}{Quantity()}


\end{description}\end{quote}

\end{fulllineitems}

\index{get\_torsion\_force\_constant() (cg\_model.cgmodel.CGModel method)@\spxentry{get\_torsion\_force\_constant()}\spxextra{cg\_model.cgmodel.CGModel method}}

\begin{fulllineitems}
\phantomsection\label{\detokenize{cg_model:cg_model.cgmodel.CGModel.get_torsion_force_constant}}\pysiglinewithargsret{\sphinxbfcode{\sphinxupquote{get\_torsion\_force\_constant}}}{\emph{torsion}}{}
Determines the correct torsion force constant for a torsion (bond angle involving four particles), given their indices within the coarse grained model
\begin{quote}\begin{description}
\item[{Parameters}] \leavevmode\begin{itemize}
\item {} 
\sphinxstyleliteralstrong{\sphinxupquote{CGModel}} (\sphinxstyleliteralemphasis{\sphinxupquote{class}}) \textendash{} CGModel() class object

\item {} 
\sphinxstyleliteralstrong{\sphinxupquote{torsion}} (\sphinxstyleliteralemphasis{\sphinxupquote{List}}\sphinxstyleliteralemphasis{\sphinxupquote{( }}\sphinxhref{https://docs.python.org/3/library/functions.html\#int}{\sphinxstyleliteralemphasis{\sphinxupquote{int}}}\sphinxstyleliteralemphasis{\sphinxupquote{ )}}) \textendash{} A list of the indices for the particles in a torsion

\end{itemize}

\item[{Returns}] \leavevmode
torsion\_force\_constant: The assigned torsion force constant for the provided particles

\item[{Return type}] \leavevmode

torsion\_force\_constant: \sphinxhref{https://docs.openmm.org/development/api-python/generated/simtk.unit.quantity.Quantity.html}{Quantity()}


\end{description}\end{quote}

\end{fulllineitems}

\index{get\_torsion\_list() (cg\_model.cgmodel.CGModel method)@\spxentry{get\_torsion\_list()}\spxextra{cg\_model.cgmodel.CGModel method}}

\begin{fulllineitems}
\phantomsection\label{\detokenize{cg_model:cg_model.cgmodel.CGModel.get_torsion_list}}\pysiglinewithargsret{\sphinxbfcode{\sphinxupquote{get\_torsion\_list}}}{}{}
Construct a list of particle indices from which to define torsions for the coarse grained model
\begin{quote}\begin{description}
\item[{Parameters}] \leavevmode
\sphinxstyleliteralstrong{\sphinxupquote{CGModel}} (\sphinxstyleliteralemphasis{\sphinxupquote{class}}) \textendash{} CGModel() class object

\item[{Returns}] \leavevmode
torsions: A list of the particle indices for the torsions in the coarse grained model

\item[{Return type}] \leavevmode
torsions: List( List( \sphinxhref{https://docs.python.org/3/library/functions.html\#int}{int}, \sphinxhref{https://docs.python.org/3/library/functions.html\#int}{int}, \sphinxhref{https://docs.python.org/3/library/functions.html\#int}{int}, \sphinxhref{https://docs.python.org/3/library/functions.html\#int}{int} ) )

\end{description}\end{quote}

\end{fulllineitems}

\index{nonbonded\_interaction\_list (cg\_model.cgmodel.CGModel attribute)@\spxentry{nonbonded\_interaction\_list}\spxextra{cg\_model.cgmodel.CGModel attribute}}

\begin{fulllineitems}
\phantomsection\label{\detokenize{cg_model:cg_model.cgmodel.CGModel.nonbonded_interaction_list}}\pysigline{\sphinxbfcode{\sphinxupquote{nonbonded\_interaction\_list}}\sphinxbfcode{\sphinxupquote{ = None}}}
Initialize new (coarse grained) particle types:

\end{fulllineitems}


\end{fulllineitems}



\section{Other coarse grained model utilities}
\label{\detokenize{cg_model:module-cg_model.cgmodel}}\label{\detokenize{cg_model:other-coarse-grained-model-utilities}}\index{cg\_model.cgmodel (module)@\spxentry{cg\_model.cgmodel}\spxextra{module}}\index{get\_parent\_bead() (in module cg\_model.cgmodel)@\spxentry{get\_parent\_bead()}\spxextra{in module cg\_model.cgmodel}}

\begin{fulllineitems}
\phantomsection\label{\detokenize{cg_model:cg_model.cgmodel.get_parent_bead}}\pysiglinewithargsret{\sphinxcode{\sphinxupquote{cg\_model.cgmodel.}}\sphinxbfcode{\sphinxupquote{get\_parent\_bead}}}{\emph{cgmodel}, \emph{monomer\_index}, \emph{bead\_index}, \emph{backbone\_bead\_index=None}, \emph{sidechain\_bead=False}}{}
Determines if a particle is bonded to any other particles (Used for coarse grained model construction.)
\begin{quote}\begin{description}
\item[{Parameters}] \leavevmode\begin{itemize}
\item {} 
\sphinxstyleliteralstrong{\sphinxupquote{cgmodel}} (\sphinxstyleliteralemphasis{\sphinxupquote{class}}) \textendash{} CGModel() class object

\item {} 
\sphinxstyleliteralstrong{\sphinxupquote{monomer\_index}} (\sphinxhref{https://docs.python.org/3/library/functions.html\#int}{\sphinxstyleliteralemphasis{\sphinxupquote{int}}}) \textendash{} Index of the monomer containing the bead we are interested in

\item {} 
\sphinxstyleliteralstrong{\sphinxupquote{bead\_index}} (\sphinxhref{https://docs.python.org/3/library/functions.html\#int}{\sphinxstyleliteralemphasis{\sphinxupquote{int}}}) \textendash{} Index of the particle we are interested in identifying bonds for

\item {} 
\sphinxstyleliteralstrong{\sphinxupquote{backbone\_bead\_index}} (\sphinxhref{https://docs.python.org/3/library/functions.html\#int}{\sphinxstyleliteralemphasis{\sphinxupquote{int}}}) \textendash{} If this bead is a backbone bead, and the monomer it belongs to contains multiple backbone beads, this will provide the position of the backbone bead

\item {} 
\sphinxstyleliteralstrong{\sphinxupquote{sidechain\_bead}} (\sphinxstyleliteralemphasis{\sphinxupquote{Logical}}) \textendash{} Indicates whether or not this bead is part of a sidechain.

\end{itemize}

\item[{Returns}] \leavevmode
parent\_bead: Index for particle(s) that the target particle is bonded to

\item[{Return type}] \leavevmode
\sphinxhref{https://docs.python.org/3/library/functions.html\#int}{int}

\end{description}\end{quote}

\end{fulllineitems}



\chapter{Thermodynamic analysis tools for coarse grained modeling}
\label{\detokenize{thermo:thermodynamic-analysis-tools-for-coarse-grained-modeling}}\label{\detokenize{thermo::doc}}
This page details the functions and classes in src/thermo


\section{Tools to calculate the heat capacity with pymbar}
\label{\detokenize{thermo:tools-to-calculate-the-heat-capacity-with-pymbar}}
Shown below are functions/tools used in order to calculate
the heat capacity with pymbar.


\chapter{Utilities for the ‘foldamers’ package}
\label{\detokenize{utilities:utilities-for-the-foldamers-package}}\label{\detokenize{utilities::doc}}
This page details the functions and classes in src/util.


\section{Input/Output options (src/utilities/iotools.py)}
\label{\detokenize{utilities:input-output-options-src-utilities-iotools-py}}
Shown below is a detailed description of the input/output
options for the foldamers package.

\phantomsection\label{\detokenize{utilities:module-utilities.iotools}}\index{utilities.iotools (module)@\spxentry{utilities.iotools}\spxextra{module}}\index{write\_bonds() (in module utilities.iotools)@\spxentry{write\_bonds()}\spxextra{in module utilities.iotools}}

\begin{fulllineitems}
\phantomsection\label{\detokenize{utilities:utilities.iotools.write_bonds}}\pysiglinewithargsret{\sphinxcode{\sphinxupquote{utilities.iotools.}}\sphinxbfcode{\sphinxupquote{write\_bonds}}}{\emph{CGModel}, \emph{pdb\_object}}{}
Writes the bonds from an input CGModel class object to the file object ‘pdb\_object’, using PDB ‘CONECT’ syntax.

CGModel: Coarse grained model class object

pdb\_object: File object to which we will write the bond list

\end{fulllineitems}

\index{write\_cg\_pdb() (in module utilities.iotools)@\spxentry{write\_cg\_pdb()}\spxextra{in module utilities.iotools}}

\begin{fulllineitems}
\phantomsection\label{\detokenize{utilities:utilities.iotools.write_cg_pdb}}\pysiglinewithargsret{\sphinxcode{\sphinxupquote{utilities.iotools.}}\sphinxbfcode{\sphinxupquote{write\_cg\_pdb}}}{\emph{cgmodel}, \emph{file\_name}}{}
Writes the positions from an input CGModel class object to the file ‘filename’.  Used to test the compatibility of coarse grained model parameters with the OpenMM PDBFile() functions, which are needed to write coordinates to a PDB file during MD simulations.

CGModel: Coarse grained model class object

filename: Path to the file where we will write PDB coordinates.

\end{fulllineitems}

\index{write\_pdbfile\_without\_topology() (in module utilities.iotools)@\spxentry{write\_pdbfile\_without\_topology()}\spxextra{in module utilities.iotools}}

\begin{fulllineitems}
\phantomsection\label{\detokenize{utilities:utilities.iotools.write_pdbfile_without_topology}}\pysiglinewithargsret{\sphinxcode{\sphinxupquote{utilities.iotools.}}\sphinxbfcode{\sphinxupquote{write\_pdbfile\_without\_topology}}}{\emph{CGModel}, \emph{filename}, \emph{energy=None}}{}
Writes the positions from an input CGModel class object to the file ‘filename’.

CGModel: Coarse grained model class object

filename: Path to the file where we will write PDB coordinates.

energy: Energy to write to the PDB file, default = None

\end{fulllineitems}



\section{Utilities and random functions (src/utilities/util.py)}
\label{\detokenize{utilities:module-utilities.util}}\label{\detokenize{utilities:utilities-and-random-functions-src-utilities-util-py}}\index{utilities.util (module)@\spxentry{utilities.util}\spxextra{module}}\index{assign\_position() (in module utilities.util)@\spxentry{assign\_position()}\spxextra{in module utilities.util}}

\begin{fulllineitems}
\phantomsection\label{\detokenize{utilities:utilities.util.assign_position}}\pysiglinewithargsret{\sphinxcode{\sphinxupquote{utilities.util.}}\sphinxbfcode{\sphinxupquote{assign\_position}}}{\emph{positions}, \emph{bond\_length}, \emph{sigma}, \emph{bead\_index}, \emph{parent\_index}}{}
Assign random position for a bead

positions: Positions for all beads in the coarse-grained model.
( np.array( num\_beads x 3 ) )

bond\_length: Bond length for all beads that are bonded,
( float * simtk.unit.distance )
default = 1.0 * unit.angstrom

positions: Positions for all beads in the coarse-grained model.
( np.array( num\_beads x 3 ) )

\end{fulllineitems}

\index{assign\_position\_lattice\_style() (in module utilities.util)@\spxentry{assign\_position\_lattice\_style()}\spxextra{in module utilities.util}}

\begin{fulllineitems}
\phantomsection\label{\detokenize{utilities:utilities.util.assign_position_lattice_style}}\pysiglinewithargsret{\sphinxcode{\sphinxupquote{utilities.util.}}\sphinxbfcode{\sphinxupquote{assign\_position\_lattice\_style}}}{\emph{cgmodel}, \emph{positions}, \emph{distance\_cutoff}, \emph{parent\_bead\_index}, \emph{bead\_index}}{}
Assign random position for a bead

positions: Positions for all beads in the coarse-grained model.
( np.array( num\_beads x 3 ) )

bond\_length: Bond length for all beads that are bonded,
( float * simtk.unit.distance )
default = 1.0 * unit.angstrom

positions: Positions for all beads in the coarse-grained model.
( np.array( num\_beads x 3 ) )

\end{fulllineitems}

\index{attempt\_lattice\_move() (in module utilities.util)@\spxentry{attempt\_lattice\_move()}\spxextra{in module utilities.util}}

\begin{fulllineitems}
\phantomsection\label{\detokenize{utilities:utilities.util.attempt_lattice_move}}\pysiglinewithargsret{\sphinxcode{\sphinxupquote{utilities.util.}}\sphinxbfcode{\sphinxupquote{attempt\_lattice\_move}}}{\emph{parent\_coordinates}, \emph{bond\_length}, \emph{move\_direction\_list}}{}
Given a set of cartesian coordinates, assign a new particle
a distance of ‘bond\_length’ away in a random direction.

parent\_coordinates: Positions for a single particle,
away from which we will place a new particle a distance
of ‘bond\_length’ away.
( np.array( float * unit.angstrom ( length = 3 ) ) )

bond\_length: Bond length for all beads that are bonded,
( float * simtk.unit.distance )
default = 1.0 * unit.angstrom

trial\_coordinates: Positions for a new trial particle
( np.array( float * unit.angstrom ( length = 3 ) ) )

\end{fulllineitems}

\index{attempt\_move() (in module utilities.util)@\spxentry{attempt\_move()}\spxextra{in module utilities.util}}

\begin{fulllineitems}
\phantomsection\label{\detokenize{utilities:utilities.util.attempt_move}}\pysiglinewithargsret{\sphinxcode{\sphinxupquote{utilities.util.}}\sphinxbfcode{\sphinxupquote{attempt\_move}}}{\emph{parent\_coordinates}, \emph{bond\_length}}{}
Given a set of cartesian coordinates, assign a new particle
a distance of ‘bond\_length’ away in a random direction.

parent\_coordinates: Positions for a single particle,
away from which we will place a new particle a distance
of ‘bond\_length’ away.
( np.array( float * unit.angstrom ( length = 3 ) ) )

bond\_length: Bond length for all beads that are bonded,
( float * simtk.unit.distance )
default = 1.0 * unit.angstrom

trial\_coordinates: Positions for a new trial particle
( np.array( float * unit.angstrom ( length = 3 ) ) )

\end{fulllineitems}

\index{collisions() (in module utilities.util)@\spxentry{collisions()}\spxextra{in module utilities.util}}

\begin{fulllineitems}
\phantomsection\label{\detokenize{utilities:utilities.util.collisions}}\pysiglinewithargsret{\sphinxcode{\sphinxupquote{utilities.util.}}\sphinxbfcode{\sphinxupquote{collisions}}}{\emph{positions}, \emph{distance\_list}, \emph{distance\_cutoff}}{}
Determine whether there are any collisions between non-bonded
particles, where a “collision” is defined as a distance shorter
than the user-provided ‘bond\_length’.

distances: List of the distances between all nonbonded particles.
( list ( float * simtk.unit.distance ( length = \# nonbonded\_interactions ) ) )

bond\_length: Bond length for all beads that are bonded,
( float * simtk.unit.distance )
default = 1.0 * unit.angstrom

collision: Logical variable stating whether or not the model has
bead collisions.
default = False

\end{fulllineitems}

\index{distance() (in module utilities.util)@\spxentry{distance()}\spxextra{in module utilities.util}}

\begin{fulllineitems}
\phantomsection\label{\detokenize{utilities:utilities.util.distance}}\pysiglinewithargsret{\sphinxcode{\sphinxupquote{utilities.util.}}\sphinxbfcode{\sphinxupquote{distance}}}{\emph{positions\_1}, \emph{positions\_2}}{}
Construct a matrix of the distances between all particles.

positions\_1: Positions for a particle
( np.array( length = 3 ) )

positions\_2: Positions for a particle
( np.array( length = 3 ) )

distance
( float * unit )

\end{fulllineitems}

\index{distance\_matrix() (in module utilities.util)@\spxentry{distance\_matrix()}\spxextra{in module utilities.util}}

\begin{fulllineitems}
\phantomsection\label{\detokenize{utilities:utilities.util.distance_matrix}}\pysiglinewithargsret{\sphinxcode{\sphinxupquote{utilities.util.}}\sphinxbfcode{\sphinxupquote{distance\_matrix}}}{\emph{positions}}{}
Construct a matrix of the distances between all particles.

positions: Positions for an array of particles.
( np.array( num\_particles x 3 ) )

distance\_matrix: Matrix containing the distances between all beads.
( np.array( num\_particles x 3 ) )

\end{fulllineitems}

\index{distances() (in module utilities.util)@\spxentry{distances()}\spxextra{in module utilities.util}}

\begin{fulllineitems}
\phantomsection\label{\detokenize{utilities:utilities.util.distances}}\pysiglinewithargsret{\sphinxcode{\sphinxupquote{utilities.util.}}\sphinxbfcode{\sphinxupquote{distances}}}{\emph{interaction\_list}, \emph{positions}}{}
Calculate the distances between a trial particle (‘new\_coordinates’)
and all existing particles (‘existing\_coordinates’).

new\_coordinates: Positions for a single trial particle
( np.array( float * unit.angstrom ( length = 3 ) ) )

existing\_coordinates: Positions for a single trial particle
( np.array( float * unit.angstrom ( shape = num\_particles x 3 ) ) )

distances: List of the distances between all nonbonded particles.
( list ( float * simtk.unit.distance ( length = \# nonbonded\_interactions ) ) )

\end{fulllineitems}

\index{first\_bead() (in module utilities.util)@\spxentry{first\_bead()}\spxextra{in module utilities.util}}

\begin{fulllineitems}
\phantomsection\label{\detokenize{utilities:utilities.util.first_bead}}\pysiglinewithargsret{\sphinxcode{\sphinxupquote{utilities.util.}}\sphinxbfcode{\sphinxupquote{first\_bead}}}{\emph{positions}}{}
Determine if we have any particles in ‘positions’
Parameters
———-
positions: Positions for all beads in the coarse-grained model.
( np.array( float * unit ( shape = num\_beads x 3 ) ) )
Returns
——-
first\_bead: Logical variable stating if this is the first particle.

\end{fulllineitems}

\index{get\_move() (in module utilities.util)@\spxentry{get\_move()}\spxextra{in module utilities.util}}

\begin{fulllineitems}
\phantomsection\label{\detokenize{utilities:utilities.util.get_move}}\pysiglinewithargsret{\sphinxcode{\sphinxupquote{utilities.util.}}\sphinxbfcode{\sphinxupquote{get\_move}}}{\emph{trial\_coordinates}, \emph{move\_direction}, \emph{distance}, \emph{bond\_length}, \emph{finish\_bond=False}}{}
Given a ‘move\_direction’, a current distance, and a
target ‘bond\_length’ ( Index denoting x,y,z Cartesian 
direction), update the coordinates for the particle.

trial\_coordinates: positions for a particle
( np.array( float * unit.angstrom ( length = 3 ) ) )

move\_direction: Cartesian direction in which we will
attempt a particle placement, where: x=0, y=1, z=2. 
( integer )

distance: Current distance from parent particle
( float * simtk.unit.distance )

bond\_length: Target bond\_length for particle placement.
( float * simtk.unit.distance )

finish\_bond: Logical variable determining how we will
update the coordinates for this particle.

trial\_coordinates: Updated positions for the particle
( np.array( float * unit.angstrom ( length = 3 ) ) )

\end{fulllineitems}

\index{get\_structure\_from\_library() (in module utilities.util)@\spxentry{get\_structure\_from\_library()}\spxextra{in module utilities.util}}

\begin{fulllineitems}
\phantomsection\label{\detokenize{utilities:utilities.util.get_structure_from_library}}\pysiglinewithargsret{\sphinxcode{\sphinxupquote{utilities.util.}}\sphinxbfcode{\sphinxupquote{get\_structure\_from\_library}}}{\emph{cgmodel}, \emph{high\_energy=False}, \emph{low\_energy=False}}{}
Given a coarse grained model class object, this function retrieves
a set of positions for the model from the ensemble library, in:
‘../foldamers/ensembles/\$\{backbone\_length\}\_\$\{sidechain\_length\}\_\$\{sidechain\_positions\}’
If this coarse grained model does not have an ensemble library, an 
error message will be returned and we will attempt to assign 
positions at random with ‘random\_positions()’.

cgmodel: CGModel() class object.
\begin{quote}\begin{description}
\item[{Parameters}] \leavevmode\begin{itemize}
\item {} 
\sphinxstyleliteralstrong{\sphinxupquote{high\_energy}} (\sphinxstyleliteralemphasis{\sphinxupquote{Logical}}) \textendash{} If set to ‘True’, this function will generate an ensemble of high-energy structures, default = False

\item {} 
\sphinxstyleliteralstrong{\sphinxupquote{low\_energy}} (\sphinxstyleliteralemphasis{\sphinxupquote{Logical}}) \textendash{} If set to ‘True’, this function will generate an ensemble of low-energy structures, default = False

\end{itemize}

\end{description}\end{quote}

positions: Positions for all beads in the coarse-grained model.
( np.array( num\_beads x 3 ) )

\end{fulllineitems}

\index{random\_positions() (in module utilities.util)@\spxentry{random\_positions()}\spxextra{in module utilities.util}}

\begin{fulllineitems}
\phantomsection\label{\detokenize{utilities:utilities.util.random_positions}}\pysiglinewithargsret{\sphinxcode{\sphinxupquote{utilities.util.}}\sphinxbfcode{\sphinxupquote{random\_positions}}}{\emph{cgmodel}, \emph{max\_attempts=1000}, \emph{use\_library=False}, \emph{high\_energy=False}, \emph{low\_energy=False}, \emph{generate\_library=False}}{}
Assign random positions for all beads in a coarse-grained polymer.

cgmodel: CGModel() class object.

max\_attempts: The maximum number of times that we will attempt to build
a coarse grained model with the settings in ‘cgmodel’.
default = 1000

use\_library: A logical variable determining if we will generate a new
random structure, or take a random structure from the library in the following path:
‘../foldamers/ensembles/\$\{backbone\_length\}\_\$\{sidechain\_length\}\_\$\{sidechain\_positions\}’
default = True
( NOTE: By default, if use\_library = False, new structures will be added to the
\begin{quote}

ensemble library for the relevant coarse grained model.  If that model does not
have an ensemble library, one will be created. )
\end{quote}
\begin{quote}\begin{description}
\item[{Parameters}] \leavevmode\begin{itemize}
\item {} 
\sphinxstyleliteralstrong{\sphinxupquote{high\_energy}} (\sphinxstyleliteralemphasis{\sphinxupquote{Logical}}) \textendash{} If set to ‘True’, this function will generate an ensemble of high-energy structures, default = False

\item {} 
\sphinxstyleliteralstrong{\sphinxupquote{low\_energy}} (\sphinxstyleliteralemphasis{\sphinxupquote{Logical}}) \textendash{} If set to ‘True’, this function will generate an ensemble of low-energy structures, default = False

\end{itemize}

\end{description}\end{quote}

positions: Positions for all beads in the coarse-grained model.
( np.array( num\_beads x 3 ) )

\end{fulllineitems}

\index{random\_sign() (in module utilities.util)@\spxentry{random\_sign()}\spxextra{in module utilities.util}}

\begin{fulllineitems}
\phantomsection\label{\detokenize{utilities:utilities.util.random_sign}}\pysiglinewithargsret{\sphinxcode{\sphinxupquote{utilities.util.}}\sphinxbfcode{\sphinxupquote{random\_sign}}}{\emph{number}}{}
Returns ‘number’ with a random sign.

number: float

number

\end{fulllineitems}



\chapter{Indices and tables}
\label{\detokenize{index:indices-and-tables}}\begin{itemize}
\item {} 
\DUrole{xref,std,std-ref}{genindex}

\item {} 
\DUrole{xref,std,std-ref}{modindex}

\item {} 
\DUrole{xref,std,std-ref}{search}

\end{itemize}


\renewcommand{\indexname}{Python Module Index}
\begin{sphinxtheindex}
\let\bigletter\sphinxstyleindexlettergroup
\bigletter{c}
\item\relax\sphinxstyleindexentry{cg\_model.cgmodel}\sphinxstyleindexpageref{cg_model:\detokenize{module-cg_model.cgmodel}}
\indexspace
\bigletter{u}
\item\relax\sphinxstyleindexentry{utilities.iotools}\sphinxstyleindexpageref{utilities:\detokenize{module-utilities.iotools}}
\item\relax\sphinxstyleindexentry{utilities.util}\sphinxstyleindexpageref{utilities:\detokenize{module-utilities.util}}
\end{sphinxtheindex}

\renewcommand{\indexname}{Index}
\printindex
\end{document}