%% Generated by Sphinx.
\def\sphinxdocclass{report}
\documentclass[letterpaper,12pt,english,openany,oneside]{sphinxmanual}
\ifdefined\pdfpxdimen
   \let\sphinxpxdimen\pdfpxdimen\else\newdimen\sphinxpxdimen
\fi \sphinxpxdimen=.75bp\relax

\PassOptionsToPackage{warn}{textcomp}
\usepackage[utf8]{inputenc}
\ifdefined\DeclareUnicodeCharacter
% support both utf8 and utf8x syntaxes
  \ifdefined\DeclareUnicodeCharacterAsOptional
    \def\sphinxDUC#1{\DeclareUnicodeCharacter{"#1}}
  \else
    \let\sphinxDUC\DeclareUnicodeCharacter
  \fi
  \sphinxDUC{00A0}{\nobreakspace}
  \sphinxDUC{2500}{\sphinxunichar{2500}}
  \sphinxDUC{2502}{\sphinxunichar{2502}}
  \sphinxDUC{2514}{\sphinxunichar{2514}}
  \sphinxDUC{251C}{\sphinxunichar{251C}}
  \sphinxDUC{2572}{\textbackslash}
\fi
\usepackage{cmap}
\usepackage[T1]{fontenc}
\usepackage{amsmath,amssymb,amstext}
\usepackage{babel}



\usepackage{times}
\expandafter\ifx\csname T@LGR\endcsname\relax
\else
% LGR was declared as font encoding
  \substitutefont{LGR}{\rmdefault}{cmr}
  \substitutefont{LGR}{\sfdefault}{cmss}
  \substitutefont{LGR}{\ttdefault}{cmtt}
\fi
\expandafter\ifx\csname T@X2\endcsname\relax
  \expandafter\ifx\csname T@T2A\endcsname\relax
  \else
  % T2A was declared as font encoding
    \substitutefont{T2A}{\rmdefault}{cmr}
    \substitutefont{T2A}{\sfdefault}{cmss}
    \substitutefont{T2A}{\ttdefault}{cmtt}
  \fi
\else
% X2 was declared as font encoding
  \substitutefont{X2}{\rmdefault}{cmr}
  \substitutefont{X2}{\sfdefault}{cmss}
  \substitutefont{X2}{\ttdefault}{cmtt}
\fi


\usepackage[Bjarne]{fncychap}
\usepackage{sphinx}

\fvset{fontsize=\small}
\usepackage{geometry}

% Include hyperref last.
\usepackage{hyperref}
% Fix anchor placement for figures with captions.
\usepackage{hypcap}% it must be loaded after hyperref.
% Set up styles of URL: it should be placed after hyperref.
\urlstyle{same}

\usepackage{sphinxmessages}
\setcounter{tocdepth}{1}



\title{foldamers Documentation}
\date{May 12, 2019}
\release{0.0}
\author{Shirts research group\\ \\Garrett A. Meek\\Lenny T. Fobe\\Michael R. Shirts\\ \\Dept. of Chemical and Biological Engineering\\University of Colorado Boulder}
\newcommand{\sphinxlogo}{\vbox{}}
\renewcommand{\releasename}{Release}
\makeindex
\begin{document}

\pagestyle{empty}
\sphinxmaketitle
\pagestyle{plain}
\sphinxtableofcontents
\pagestyle{normal}
\phantomsection\label{\detokenize{index::doc}}


This documentation is generated automatically using Sphinx, which reads all docstring-formatted comments from Python functions in the ‘foldamers’ repository.  (See foldamers/doc for Sphinx source files.)


\chapter{Coarse grained model utilities}
\label{\detokenize{cg_model:coarse-grained-model-utilities}}\label{\detokenize{cg_model::doc}}
This page details the functions and classes in src/cg\_model/cgmodel.py


\section{‘CGModel’ class for OpenMM simulation}
\label{\detokenize{cg_model:cgmodel-class-for-openmm-simulation}}
Shown below is a detailed description of the ‘cgmodel’ class object,
which contains all information about a coarse grained model.

\phantomsection\label{\detokenize{cg_model:module-cg_model.cgmodel}}\index{cg\_model.cgmodel (module)@\spxentry{cg\_model.cgmodel}\spxextra{module}}

\section{Other coarse grained model utilities}
\label{\detokenize{cg_model:module-cg_model.cgmodel}}\label{\detokenize{cg_model:other-coarse-grained-model-utilities}}\index{cg\_model.cgmodel (module)@\spxentry{cg\_model.cgmodel}\spxextra{module}}\index{CGModel (class in cg\_model.cgmodel)@\spxentry{CGModel}\spxextra{class in cg\_model.cgmodel}}

\begin{fulllineitems}
\phantomsection\label{\detokenize{cg_model:cg_model.cgmodel.CGModel}}\pysiglinewithargsret{\sphinxbfcode{\sphinxupquote{class }}\sphinxcode{\sphinxupquote{cg\_model.cgmodel.}}\sphinxbfcode{\sphinxupquote{CGModel}}}{\emph{positions=None, polymer\_length=12, backbone\_length=1, sidechain\_length=1, sidechain\_positions={[}0{]}, masses=Quantity(value=12.0, unit=dalton), sigma=Quantity(value=8.4, unit=angstrom), epsilon=Quantity(value=0.5, unit=kilocalorie/mole), bond\_lengths=Quantity(value=1.0, unit=angstrom), bond\_force\_constants=990000.0, charges=Quantity(value=0.0, unit=elementary charge), constrain\_bonds=False, include\_bond\_forces=True, include\_nonbonded\_forces=True, include\_bond\_angle\_forces=True, include\_torsion\_forces=True, check\_energy\_conservation=True}}{}
Construct a coarse grained model.

positions: Positions for all of the particles, default = None

polymer\_length: Number of monomer units (integer), default = 8

backbone\_length: Number of beads in the backbone 
portion of each (individual) monomer (integer), default = 1

sidechain\_length: Number of beads in the sidechain
portion of each (individual) monomer (integer), default = 1

sidechain\_positions: List of integers defining the backbone
bead indices upon which we will place the sidechains,
default = {[}0{]} (Place a sidechain on the backbone bead with
index “0” (first backbone bead) in each (individual) monomer

masses: Masses of all particle types
( List ( {[} {[} Backbone masses {]}, {[} Sidechain masses {]} {]} ) )
default = {[} {[} 12.0 * unit.amu {]}, {[} 12.0 * unit.amu {]} {]}

sigma: Non-bonded bead Lennard-Jones equilibrium interaction distance
( float * simtk.unit.distance )
default = 8.4 * unit.angstrom

epsilon: Non-bonded Lennard-Jones equilibrium interaction strength
( float * simtk.unit.energy )
default = 0.5 * unit.kilocalorie\_per\_mole

bond\_lengths: Bond lengths for all bond types
( float * simtk.unit.distance )
default = 1.0 * unit.angstrom

bond\_force\_constants: Bond force constants for all bond types
( float )
default = 9.9e5 kJ/mol/A\textasciicircum{}2

charges: Charges for all beads
( float * simtk.unit.charge )
default = 0.0 * unit.elementary\_charge (for all beads)

num\_beads: Total number of particles in the coarse grained model
( integer )
default = polymer\_length * ( backbone\_length + sidechain\_length )

system: OpenMM system object, which stores forces, and can be used
to check a model for energy conservation
( OpenMM System() class object )
default = None

topology: OpenMM topology object, which stores bonds, angles, and
other structural attributes of the coarse grained model
( OpenMM Topology() class object )
default = None

constrain\_bonds: Logical variable determining whether bond constraints
are applied during a molecular dynamics simulation of the system.
( Logical )
default = False

bond\_list: List of bonds in the coarse grained model
( List( {[} {[} int, int {]} for \# bonds {]} ) )
default = None

angle\_list: List of bond angles that are defined for this coarse
grained model
( List( {[} {[} int, int, int {]} for \# bond angles {]} ) )

torsion\_list: List of torsions that are defined for this coarse
grained model
List( {[} {[} int, int, int, int {]} for \# torsions {]} ) )

include\_bond\_forces: Include contributions from bond
(harmonic) potentials when calculating the potential energy
( Logical )
default = True

include\_nonbonded\_forces: Include contributions from nonbonded
interactions when calculating the potential energy
( Logical )
default = True

include\_bond\_angle\_forces: Include contributions from bond angles
when calculating the potential energy
( Logical )
default = False

include\_torsion\_forces: Include contributions from torsions
when calculating the potential energy
( Logical )
default = False

polymer\_length
backbone\_length
sidechain\_length
sidechain\_positions
masses
sigma
epsilon
bond\_lengths
bond\_force\_constants
charges
num\_beads
positions
system
topology
constrain\_bonds
bond\_list
angle\_list
torsion\_list
include\_bond\_forces
include\_nonbonded\_forces
include\_bond\_angle\_forces
include\_torsion\_forces
\index{charges (cg\_model.cgmodel.CGModel attribute)@\spxentry{charges}\spxextra{cg\_model.cgmodel.CGModel attribute}}

\begin{fulllineitems}
\phantomsection\label{\detokenize{cg_model:cg_model.cgmodel.CGModel.charges}}\pysigline{\sphinxbfcode{\sphinxupquote{charges}}\sphinxbfcode{\sphinxupquote{ = None}}}
Get bond, angle, and torsion lists.

\end{fulllineitems}

\index{constrain\_bonds (cg\_model.cgmodel.CGModel attribute)@\spxentry{constrain\_bonds}\spxextra{cg\_model.cgmodel.CGModel attribute}}

\begin{fulllineitems}
\phantomsection\label{\detokenize{cg_model:cg_model.cgmodel.CGModel.constrain_bonds}}\pysigline{\sphinxbfcode{\sphinxupquote{constrain\_bonds}}\sphinxbfcode{\sphinxupquote{ = None}}}
Make a list of coarse grained particle masses:

\end{fulllineitems}

\index{get\_bond\_angle\_list() (cg\_model.cgmodel.CGModel method)@\spxentry{get\_bond\_angle\_list()}\spxextra{cg\_model.cgmodel.CGModel method}}

\begin{fulllineitems}
\phantomsection\label{\detokenize{cg_model:cg_model.cgmodel.CGModel.get_bond_angle_list}}\pysiglinewithargsret{\sphinxbfcode{\sphinxupquote{get\_bond\_angle\_list}}}{}{}
Construct a list of bond angles for our coarse grained model

\end{fulllineitems}

\index{get\_bond\_list() (cg\_model.cgmodel.CGModel method)@\spxentry{get\_bond\_list()}\spxextra{cg\_model.cgmodel.CGModel method}}

\begin{fulllineitems}
\phantomsection\label{\detokenize{cg_model:cg_model.cgmodel.CGModel.get_bond_list}}\pysiglinewithargsret{\sphinxbfcode{\sphinxupquote{get\_bond\_list}}}{}{}
Construct a bond list for the coarse grained model

\end{fulllineitems}

\index{get\_nonbonded\_interaction\_list() (cg\_model.cgmodel.CGModel method)@\spxentry{get\_nonbonded\_interaction\_list()}\spxextra{cg\_model.cgmodel.CGModel method}}

\begin{fulllineitems}
\phantomsection\label{\detokenize{cg_model:cg_model.cgmodel.CGModel.get_nonbonded_interaction_list}}\pysiglinewithargsret{\sphinxbfcode{\sphinxupquote{get\_nonbonded\_interaction\_list}}}{}{}
Construct a nonbonded interaction list for our coarse grained model

\end{fulllineitems}

\index{get\_torsion\_list() (cg\_model.cgmodel.CGModel method)@\spxentry{get\_torsion\_list()}\spxextra{cg\_model.cgmodel.CGModel method}}

\begin{fulllineitems}
\phantomsection\label{\detokenize{cg_model:cg_model.cgmodel.CGModel.get_torsion_list}}\pysiglinewithargsret{\sphinxbfcode{\sphinxupquote{get\_torsion\_list}}}{}{}
Construct a torsion list for our coarse grained model

\end{fulllineitems}


\end{fulllineitems}

\index{add\_new\_elements() (in module cg\_model.cgmodel)@\spxentry{add\_new\_elements()}\spxextra{in module cg\_model.cgmodel}}

\begin{fulllineitems}
\phantomsection\label{\detokenize{cg_model:cg_model.cgmodel.add_new_elements}}\pysiglinewithargsret{\sphinxcode{\sphinxupquote{cg\_model.cgmodel.}}\sphinxbfcode{\sphinxupquote{add\_new\_elements}}}{\emph{cgmodel}, \emph{list\_of\_masses}}{}
Adds new coarse grained particle types to OpenMM

cgmodel: CGModel() class object

list\_of\_masses: List of masses for the particles we want to add to OpenMM

\end{fulllineitems}

\index{build\_system() (in module cg\_model.cgmodel)@\spxentry{build\_system()}\spxextra{in module cg\_model.cgmodel}}

\begin{fulllineitems}
\phantomsection\label{\detokenize{cg_model:cg_model.cgmodel.build_system}}\pysiglinewithargsret{\sphinxcode{\sphinxupquote{cg\_model.cgmodel.}}\sphinxbfcode{\sphinxupquote{build\_system}}}{\emph{cgmodel}}{}
Builds an OpenMM System() class object, given a CGModel() class object as input.

cgmodel: CGModel() class object

system: OpenMM System() class object

\end{fulllineitems}

\index{get\_parent\_bead() (in module cg\_model.cgmodel)@\spxentry{get\_parent\_bead()}\spxextra{in module cg\_model.cgmodel}}

\begin{fulllineitems}
\phantomsection\label{\detokenize{cg_model:cg_model.cgmodel.get_parent_bead}}\pysiglinewithargsret{\sphinxcode{\sphinxupquote{cg\_model.cgmodel.}}\sphinxbfcode{\sphinxupquote{get\_parent\_bead}}}{\emph{cgmodel}, \emph{bead\_index}, \emph{backbone\_bead\_index=None}, \emph{sidechain\_bead=False}}{}
Determines the particle to which a given particle is bonded.  (Used for coarse grained model construction.)

cgmodel: CGModel() class object

bead\_index: Index of the particle for which we would like to determine the parent particle it is bonded to.
( integer )
Default = None

backbone\_bead\_index: If this bead is a backbone bead, this index tells us its index (within a monomer) along the backbone
( integer )
Default = None

sidechain\_bead: Logical variable stating whether or not this bead is in the sidechain.
( Logical )
Default = False

parent\_bead: Index for the particle that ‘bead\_index’ is bonded to.
( Integer )

\end{fulllineitems}

\index{get\_particle\_masses() (in module cg\_model.cgmodel)@\spxentry{get\_particle\_masses()}\spxextra{in module cg\_model.cgmodel}}

\begin{fulllineitems}
\phantomsection\label{\detokenize{cg_model:cg_model.cgmodel.get_particle_masses}}\pysiglinewithargsret{\sphinxcode{\sphinxupquote{cg\_model.cgmodel.}}\sphinxbfcode{\sphinxupquote{get\_particle\_masses}}}{\emph{cgmodel}}{}
Returns a list of unique particle masses

cgmodel: CGModel() class object

List( unique particle masses )

\end{fulllineitems}



\chapter{Thermodynamic analysis tools for coarse grained modeling}
\label{\detokenize{thermo:thermodynamic-analysis-tools-for-coarse-grained-modeling}}\label{\detokenize{thermo::doc}}
This page details the functions and classes in src/thermo


\section{Tools to calculate the heat capacity with pymbar}
\label{\detokenize{thermo:tools-to-calculate-the-heat-capacity-with-pymbar}}
Shown below are functions/tools used in order to calculate
the heat capacity with pymbar.


\chapter{Utilities for the ‘foldamers’ package}
\label{\detokenize{utilities:utilities-for-the-foldamers-package}}\label{\detokenize{utilities::doc}}
This page details the functions and classes in src/util.


\section{Input/Output options (src/utilities/iotools.py)}
\label{\detokenize{utilities:input-output-options-src-utilities-iotools-py}}
Shown below is a detailed description of the input/output
options for the foldamers package.

\phantomsection\label{\detokenize{utilities:module-utilities.iotools}}\index{utilities.iotools (module)@\spxentry{utilities.iotools}\spxextra{module}}\index{write\_pdbfile\_without\_topology() (in module utilities.iotools)@\spxentry{write\_pdbfile\_without\_topology()}\spxextra{in module utilities.iotools}}

\begin{fulllineitems}
\phantomsection\label{\detokenize{utilities:utilities.iotools.write_pdbfile_without_topology}}\pysiglinewithargsret{\sphinxcode{\sphinxupquote{utilities.iotools.}}\sphinxbfcode{\sphinxupquote{write\_pdbfile\_without\_topology}}}{\emph{CGModel}, \emph{filename}}{}
Writes the positions in ‘CGModel’ to the file ‘filename’.

CGModel: Coarse grained model class object

filename: Path to the file where we will write PDB coordinates.

\end{fulllineitems}



\section{Utilities and random functions (src/utilities/util.py)}
\label{\detokenize{utilities:module-utilities.util}}\label{\detokenize{utilities:utilities-and-random-functions-src-utilities-util-py}}\index{utilities.util (module)@\spxentry{utilities.util}\spxextra{module}}\index{assign\_position() (in module utilities.util)@\spxentry{assign\_position()}\spxextra{in module utilities.util}}

\begin{fulllineitems}
\phantomsection\label{\detokenize{utilities:utilities.util.assign_position}}\pysiglinewithargsret{\sphinxcode{\sphinxupquote{utilities.util.}}\sphinxbfcode{\sphinxupquote{assign\_position}}}{\emph{positions}, \emph{bond\_length}, \emph{sigma}, \emph{bead\_index}, \emph{parent\_index}}{}
Assign random position for a bead

positions: Positions for all beads in the coarse-grained model.
( np.array( num\_beads x 3 ) )

bond\_length: Bond length for all beads that are bonded,
( float * simtk.unit.distance )
default = 1.0 * unit.angstrom

positions: Positions for all beads in the coarse-grained model.
( np.array( num\_beads x 3 ) )

\end{fulllineitems}

\index{assign\_position\_lattice\_style() (in module utilities.util)@\spxentry{assign\_position\_lattice\_style()}\spxextra{in module utilities.util}}

\begin{fulllineitems}
\phantomsection\label{\detokenize{utilities:utilities.util.assign_position_lattice_style}}\pysiglinewithargsret{\sphinxcode{\sphinxupquote{utilities.util.}}\sphinxbfcode{\sphinxupquote{assign\_position\_lattice\_style}}}{\emph{cgmodel}, \emph{positions}, \emph{distance\_cutoff}, \emph{bead\_index}, \emph{parent\_index}}{}
Assign random position for a bead

positions: Positions for all beads in the coarse-grained model.
( np.array( num\_beads x 3 ) )

bond\_length: Bond length for all beads that are bonded,
( float * simtk.unit.distance )
default = 1.0 * unit.angstrom

positions: Positions for all beads in the coarse-grained model.
( np.array( num\_beads x 3 ) )

\end{fulllineitems}

\index{attempt\_lattice\_move() (in module utilities.util)@\spxentry{attempt\_lattice\_move()}\spxextra{in module utilities.util}}

\begin{fulllineitems}
\phantomsection\label{\detokenize{utilities:utilities.util.attempt_lattice_move}}\pysiglinewithargsret{\sphinxcode{\sphinxupquote{utilities.util.}}\sphinxbfcode{\sphinxupquote{attempt\_lattice\_move}}}{\emph{parent\_coordinates}, \emph{bond\_length}, \emph{move\_direction\_list}}{}
Given a set of cartesian coordinates, assign a new particle
a distance of ‘bond\_length’ away in a random direction.

parent\_coordinates: Positions for a single particle,
away from which we will place a new particle a distance
of ‘bond\_length’ away.
( np.array( float * unit.angstrom ( length = 3 ) ) )

bond\_length: Bond length for all beads that are bonded,
( float * simtk.unit.distance )
default = 1.0 * unit.angstrom

trial\_coordinates: Positions for a new trial particle
( np.array( float * unit.angstrom ( length = 3 ) ) )

\end{fulllineitems}

\index{attempt\_move() (in module utilities.util)@\spxentry{attempt\_move()}\spxextra{in module utilities.util}}

\begin{fulllineitems}
\phantomsection\label{\detokenize{utilities:utilities.util.attempt_move}}\pysiglinewithargsret{\sphinxcode{\sphinxupquote{utilities.util.}}\sphinxbfcode{\sphinxupquote{attempt\_move}}}{\emph{parent\_coordinates}, \emph{bond\_length}}{}
Given a set of cartesian coordinates, assign a new particle
a distance of ‘bond\_length’ away in a random direction.

parent\_coordinates: Positions for a single particle,
away from which we will place a new particle a distance
of ‘bond\_length’ away.
( np.array( float * unit.angstrom ( length = 3 ) ) )

bond\_length: Bond length for all beads that are bonded,
( float * simtk.unit.distance )
default = 1.0 * unit.angstrom

trial\_coordinates: Positions for a new trial particle
( np.array( float * unit.angstrom ( length = 3 ) ) )

\end{fulllineitems}

\index{collisions() (in module utilities.util)@\spxentry{collisions()}\spxextra{in module utilities.util}}

\begin{fulllineitems}
\phantomsection\label{\detokenize{utilities:utilities.util.collisions}}\pysiglinewithargsret{\sphinxcode{\sphinxupquote{utilities.util.}}\sphinxbfcode{\sphinxupquote{collisions}}}{\emph{distance\_list}, \emph{distance\_cutoff}}{}
Determine whether there are any collisions between non-bonded
particles, where a “collision” is defined as a distance shorter
than the user-provided ‘bond\_length’.

distances: List of the distances between all nonbonded particles.
( list ( float * simtk.unit.distance ( length = \# nonbonded\_interactions ) ) )

bond\_length: Bond length for all beads that are bonded,
( float * simtk.unit.distance )
default = 1.0 * unit.angstrom

collision: Logical variable stating whether or not the model has
bead collisions.
default = False

\end{fulllineitems}

\index{distance() (in module utilities.util)@\spxentry{distance()}\spxextra{in module utilities.util}}

\begin{fulllineitems}
\phantomsection\label{\detokenize{utilities:utilities.util.distance}}\pysiglinewithargsret{\sphinxcode{\sphinxupquote{utilities.util.}}\sphinxbfcode{\sphinxupquote{distance}}}{\emph{positions\_1}, \emph{positions\_2}}{}
Construct a matrix of the distances between all particles.

positions\_1: Positions for a particle
( np.array( length = 3 ) )

positions\_2: Positions for a particle
( np.array( length = 3 ) )

distance
( float * unit )

\end{fulllineitems}

\index{distance\_matrix() (in module utilities.util)@\spxentry{distance\_matrix()}\spxextra{in module utilities.util}}

\begin{fulllineitems}
\phantomsection\label{\detokenize{utilities:utilities.util.distance_matrix}}\pysiglinewithargsret{\sphinxcode{\sphinxupquote{utilities.util.}}\sphinxbfcode{\sphinxupquote{distance\_matrix}}}{\emph{positions}}{}
Construct a matrix of the distances between all particles.

positions: Positions for an array of particles.
( np.array( num\_particles x 3 ) )

distance\_matrix: Matrix containing the distances between all beads.
( np.array( num\_particles x 3 ) )

\end{fulllineitems}

\index{distances() (in module utilities.util)@\spxentry{distances()}\spxextra{in module utilities.util}}

\begin{fulllineitems}
\phantomsection\label{\detokenize{utilities:utilities.util.distances}}\pysiglinewithargsret{\sphinxcode{\sphinxupquote{utilities.util.}}\sphinxbfcode{\sphinxupquote{distances}}}{\emph{interaction\_list}, \emph{positions}}{}
Calculate the distances between a trial particle (‘new\_coordinates’)
and all existing particles (‘existing\_coordinates’).

new\_coordinates: Positions for a single trial particle
( np.array( float * unit.angstrom ( length = 3 ) ) )

existing\_coordinates: Positions for a single trial particle
( np.array( float * unit.angstrom ( shape = num\_particles x 3 ) ) )

distances: List of the distances between all nonbonded particles.
( list ( float * simtk.unit.distance ( length = \# nonbonded\_interactions ) ) )

\end{fulllineitems}

\index{first\_bead() (in module utilities.util)@\spxentry{first\_bead()}\spxextra{in module utilities.util}}

\begin{fulllineitems}
\phantomsection\label{\detokenize{utilities:utilities.util.first_bead}}\pysiglinewithargsret{\sphinxcode{\sphinxupquote{utilities.util.}}\sphinxbfcode{\sphinxupquote{first\_bead}}}{\emph{positions}}{}
Determine if we have any particles in ‘positions’
Parameters
———-
positions: Positions for all beads in the coarse-grained model.
( np.array( float * unit ( shape = num\_beads x 3 ) ) )
Returns
——-
first\_bead: Logical variable stating if this is the first particle.

\end{fulllineitems}

\index{get\_move() (in module utilities.util)@\spxentry{get\_move()}\spxextra{in module utilities.util}}

\begin{fulllineitems}
\phantomsection\label{\detokenize{utilities:utilities.util.get_move}}\pysiglinewithargsret{\sphinxcode{\sphinxupquote{utilities.util.}}\sphinxbfcode{\sphinxupquote{get\_move}}}{\emph{trial\_coordinates}, \emph{move\_direction}, \emph{distance}, \emph{bond\_length}, \emph{finish\_bond=False}}{}
Given a ‘move\_direction’, a current distance, and a
target ‘bond\_length’ ( Index denoting x,y,z Cartesian 
direction), update the coordinates for the particle.

trial\_coordinates: positions for a particle
( np.array( float * unit.angstrom ( length = 3 ) ) )

move\_direction: Cartesian direction in which we will
attempt a particle placement, where: x=0, y=1, z=2. 
( integer )

distance: Current distance from parent particle
( float * simtk.unit.distance )

bond\_length: Target bond\_length for particle placement.
( float * simtk.unit.distance )

finish\_bond: Logical variable determining how we will
update the coordinates for this particle.

trial\_coordinates: Updated positions for the particle
( np.array( float * unit.angstrom ( length = 3 ) ) )

\end{fulllineitems}

\index{random\_positions() (in module utilities.util)@\spxentry{random\_positions()}\spxextra{in module utilities.util}}

\begin{fulllineitems}
\phantomsection\label{\detokenize{utilities:utilities.util.random_positions}}\pysiglinewithargsret{\sphinxcode{\sphinxupquote{utilities.util.}}\sphinxbfcode{\sphinxupquote{random\_positions}}}{\emph{cgmodel}, \emph{max\_attempts=100}}{}
Assign random positions for all beads in a coarse-grained polymer.

polymer\_length: Number of monomer units (integer), default = 8

backbone\_length: Number of beads in the backbone 
portion of each (individual) monomer (integer), default = 1

sidechain\_length: Number of beads in the sidechain
portion of each (individual) monomer (integer), default = 1

sidechain\_positions: List of integers defining the backbone
bead indices upon which we will place the sidechains,
default = {[}0{]} (Place a sidechain on the backbone bead with
index “0” (first backbone bead) in each (individual) monomer

bond\_length: Bond length for all beads that are bonded,
( float * simtk.unit.distance )
default = 1.0 * unit.angstrom

sigma: Non-bonded bead Lennard-Jones interaction distances,
( float * simtk.unit.distance )
default = 8.4 * unit.angstrom

bb\_bond\_length: Bond length for all bonded backbone beads,
( float * simtk.unit.distance )
default = 1.0 * unit.angstrom

bs\_bond\_length: Bond length for all backbone-sidechain bonds,
( float * simtk.unit.distance )
default = 1.0 * unit.angstrom

ss\_bond\_length: Bond length for all beads within a sidechain,
( float * simtk.unit.distance )
default = 1.0 * unit.angstrom

positions: Positions for all beads in the coarse-grained model.
( np.array( num\_beads x 3 ) )

\end{fulllineitems}

\index{random\_sign() (in module utilities.util)@\spxentry{random\_sign()}\spxextra{in module utilities.util}}

\begin{fulllineitems}
\phantomsection\label{\detokenize{utilities:utilities.util.random_sign}}\pysiglinewithargsret{\sphinxcode{\sphinxupquote{utilities.util.}}\sphinxbfcode{\sphinxupquote{random\_sign}}}{\emph{number}}{}
Returns ‘number’ with a random sign.

number: float

number

\end{fulllineitems}



\chapter{Indices and tables}
\label{\detokenize{index:indices-and-tables}}\begin{itemize}
\item {} 
\DUrole{xref,std,std-ref}{genindex}

\item {} 
\DUrole{xref,std,std-ref}{modindex}

\item {} 
\DUrole{xref,std,std-ref}{search}

\end{itemize}


\renewcommand{\indexname}{Python Module Index}
\begin{sphinxtheindex}
\let\bigletter\sphinxstyleindexlettergroup
\bigletter{c}
\item\relax\sphinxstyleindexentry{cg\_model.cgmodel}\sphinxstyleindexpageref{cg_model:\detokenize{module-cg_model.cgmodel}}
\indexspace
\bigletter{u}
\item\relax\sphinxstyleindexentry{utilities.iotools}\sphinxstyleindexpageref{utilities:\detokenize{module-utilities.iotools}}
\item\relax\sphinxstyleindexentry{utilities.util}\sphinxstyleindexpageref{utilities:\detokenize{module-utilities.util}}
\end{sphinxtheindex}

\renewcommand{\indexname}{Index}
\printindex
\end{document}