%% Generated by Sphinx.
\def\sphinxdocclass{report}
\documentclass[letterpaper,10pt,english,openany,oneside]{sphinxmanual}
\ifdefined\pdfpxdimen
   \let\sphinxpxdimen\pdfpxdimen\else\newdimen\sphinxpxdimen
\fi \sphinxpxdimen=.75bp\relax

\PassOptionsToPackage{warn}{textcomp}
\usepackage[utf8]{inputenc}
\ifdefined\DeclareUnicodeCharacter
% support both utf8 and utf8x syntaxes
\edef\sphinxdqmaybe{\ifdefined\DeclareUnicodeCharacterAsOptional\string"\fi}
  \DeclareUnicodeCharacter{\sphinxdqmaybe00A0}{\nobreakspace}
  \DeclareUnicodeCharacter{\sphinxdqmaybe2500}{\sphinxunichar{2500}}
  \DeclareUnicodeCharacter{\sphinxdqmaybe2502}{\sphinxunichar{2502}}
  \DeclareUnicodeCharacter{\sphinxdqmaybe2514}{\sphinxunichar{2514}}
  \DeclareUnicodeCharacter{\sphinxdqmaybe251C}{\sphinxunichar{251C}}
  \DeclareUnicodeCharacter{\sphinxdqmaybe2572}{\textbackslash}
\fi
\usepackage{cmap}
\usepackage[T1]{fontenc}
\usepackage{amsmath,amssymb,amstext}
\usepackage{babel}
\usepackage{times}
\usepackage[Bjarne]{fncychap}
\usepackage{sphinx}

\fvset{fontsize=\small}
\usepackage{geometry}

% Include hyperref last.
\usepackage{hyperref}
% Fix anchor placement for figures with captions.
\usepackage{hypcap}% it must be loaded after hyperref.
% Set up styles of URL: it should be placed after hyperref.
\urlstyle{same}

\addto\captionsenglish{\renewcommand{\figurename}{Fig.\@ }}
\makeatletter
\def\fnum@figure{\figurename\thefigure{}}
\makeatother
\addto\captionsenglish{\renewcommand{\tablename}{Table }}
\makeatletter
\def\fnum@table{\tablename\thetable{}}
\makeatother
\addto\captionsenglish{\renewcommand{\literalblockname}{Listing}}

\addto\captionsenglish{\renewcommand{\literalblockcontinuedname}{continued from previous page}}
\addto\captionsenglish{\renewcommand{\literalblockcontinuesname}{continues on next page}}
\addto\captionsenglish{\renewcommand{\sphinxnonalphabeticalgroupname}{Non-alphabetical}}
\addto\captionsenglish{\renewcommand{\sphinxsymbolsname}{Symbols}}
\addto\captionsenglish{\renewcommand{\sphinxnumbersname}{Numbers}}

\addto\extrasenglish{\def\pageautorefname{page}}

\setcounter{tocdepth}{1}



\title{foldamers Documentation}
\date{Apr 14, 2019}
\release{0.0.1}
\author{Garrett A. Meek, Lenny T. Fobe, Michael R. Shirts}
\newcommand{\sphinxlogo}{\vbox{}}
\renewcommand{\releasename}{Release}
\makeindex
\begin{document}

\pagestyle{empty}
\sphinxmaketitle
\pagestyle{plain}
\sphinxtableofcontents
\pagestyle{normal}
\phantomsection\label{\detokenize{index::doc}}


This documentation is generated automatically using Sphinx, which reads
all docstring-formatted comments from Python functions in
the ‘foldamers’ repository.  (See foldamers/doc for Sphinx
source files.)


\chapter{Coarse grained model utilities}
\label{\detokenize{cg_model:coarse-grained-model-utilities}}\label{\detokenize{cg_model::doc}}
This page details the functions and classes in src/cg\_model/cgmodel.py


\section{‘cgmodel’ class for OpenMM simulation}
\label{\detokenize{cg_model:cgmodel-class-for-openmm-simulation}}
Shown below is a detailed description of the ‘cgmodel’ class object,
which contains all information about a coarse grained model.


\section{Other coarse grained model utilities}
\label{\detokenize{cg_model:other-coarse-grained-model-utilities}}

\chapter{Coarse grained model parameter analysis utilities}
\label{\detokenize{parameter_analysis:coarse-grained-model-parameter-analysis-utilities}}\label{\detokenize{parameter_analysis::doc}}
This page details the modules, functions, and classes
in src/parameter\_analysis


\section{Parameter sampling protocols}
\label{\detokenize{parameter_analysis:parameter-sampling-protocols}}

\chapter{Thermodynamic analysis tools for coarse grained modeling}
\label{\detokenize{thermo:thermodynamic-analysis-tools-for-coarse-grained-modeling}}\label{\detokenize{thermo::doc}}
This page details the functions and classes in src/thermo


\section{Heat capacity example with pymbar}
\label{\detokenize{thermo:heat-capacity-example-with-pymbar}}
Shown below are functions/tools used in order to calculate
the heat capacity with pymbar.


\chapter{Utilities for the ‘foldamers’ package}
\label{\detokenize{util:utilities-for-the-foldamers-package}}\label{\detokenize{util::doc}}
This page details the functions and classes in src/util.


\section{Input/Output options (src/utilities/iotools.py)}
\label{\detokenize{util:input-output-options-src-utilities-iotools-py}}
Shown below is a detailed description of the input/output
options for the foldamers package.

\phantomsection\label{\detokenize{util:module-iotools}}\index{iotools (module)@\spxentry{iotools}\spxextra{module}}

\section{Utilities and random functions (src/utilities/util.py)}
\label{\detokenize{util:module-util}}\label{\detokenize{util:utilities-and-random-functions-src-utilities-util-py}}\index{util (module)@\spxentry{util}\spxextra{module}}\index{append\_position() (in module util)@\spxentry{append\_position()}\spxextra{in module util}}

\begin{fulllineitems}
\phantomsection\label{\detokenize{util:util.append_position}}\pysiglinewithargsret{\sphinxcode{\sphinxupquote{util.}}\sphinxbfcode{\sphinxupquote{append\_position}}}{\emph{positions}, \emph{new\_coordinate}}{}
Updates a set of input coordinates with ‘new\_coordinate’ in the
cartesian coordinate direction indexted by ‘direction’.

new\_coordinate: Cartesian coordinates for a particle
( np.array( float * unit ( length = 3 ) ) )

direction: Cartesian direction index for particle placement, 
where: x=0,y=1,z=2. 
( integer )

trial\_coordinates: Existing cartesian coordinates for the particle
we are updating.
( np.array( float * unit ( length = 3 ) ) )
Optional, default = None

trial\_coordinates: Updated coordinates for the particle.

\end{fulllineitems}

\index{assign\_backbone\_beads() (in module util)@\spxentry{assign\_backbone\_beads()}\spxextra{in module util}}

\begin{fulllineitems}
\phantomsection\label{\detokenize{util:util.assign_backbone_beads}}\pysiglinewithargsret{\sphinxcode{\sphinxupquote{util.}}\sphinxbfcode{\sphinxupquote{assign\_backbone\_beads}}}{\emph{positions}, \emph{monomer\_start}, \emph{backbone\_length}, \emph{sidechain\_length}, \emph{sidechain\_positions}, \emph{bond\_length}}{}
Assign random position for a backbone bead

positions: Positions for all beads in the coarse-grained model.
( np.array( num\_beads x 3 ) )

monomer\_start: Index of the bead to which we will bond this
new backbone bead.
( integer )

backbone\_length: Number of beads in the backbone
portion of each (individual) monomer (integer), default = 1

sidechain\_length: Number of beads in the sidechain
portion of each (individual) monomer (integer), default = 1

sidechain\_positions: List of integers defining the backbone
bead indices upon which we will place the sidechains,
default = {[}0{]} (Place a sidechain on the backbone bead with
index “0” (first backbone bead) in each (individual) monomer

bond\_length: Bond length for all beads that are bonded,
( float * simtk.unit.distance )
default = 1.0 * unit.angstrom

positions: Positions for all beads in the coarse-grained model.
( np.array( num\_beads x 3 ) )

\end{fulllineitems}

\index{assign\_position() (in module util)@\spxentry{assign\_position()}\spxextra{in module util}}

\begin{fulllineitems}
\phantomsection\label{\detokenize{util:util.assign_position}}\pysiglinewithargsret{\sphinxcode{\sphinxupquote{util.}}\sphinxbfcode{\sphinxupquote{assign\_position}}}{\emph{positions}, \emph{bond\_length}, \emph{parent\_index=-1}}{}
Assign random position for a bead

positions: Positions for all beads in the coarse-grained model.
( np.array( num\_beads x 3 ) )

bond\_length: Bond length for all beads that are bonded,
( float * simtk.unit.distance )
default = 1.0 * unit.angstrom

positions: Positions for all beads in the coarse-grained model.
( np.array( num\_beads x 3 ) )

\end{fulllineitems}

\index{assign\_sidechain\_beads() (in module util)@\spxentry{assign\_sidechain\_beads()}\spxextra{in module util}}

\begin{fulllineitems}
\phantomsection\label{\detokenize{util:util.assign_sidechain_beads}}\pysiglinewithargsret{\sphinxcode{\sphinxupquote{util.}}\sphinxbfcode{\sphinxupquote{assign\_sidechain\_beads}}}{\emph{positions}, \emph{sidechain\_length}, \emph{bond\_length}}{}
Assign random position for all sidechain beads

positions: Positions for all beads in the coarse-grained model.
( np.array( num\_beads x 3 ) )

sidechain\_length: Number of beads in the sidechain
portion of each (individual) monomer (integer), default = 1

bond\_length: Bond length for all beads that are bonded,
( float * simtk.unit.distance )
default = 1.0 * unit.angstrom

positions: Positions for all beads in the coarse-grained model.
( np.array( num\_beads x 3 ) )

\end{fulllineitems}

\index{attempt\_move() (in module util)@\spxentry{attempt\_move()}\spxextra{in module util}}

\begin{fulllineitems}
\phantomsection\label{\detokenize{util:util.attempt_move}}\pysiglinewithargsret{\sphinxcode{\sphinxupquote{util.}}\sphinxbfcode{\sphinxupquote{attempt\_move}}}{\emph{parent\_coordinates}, \emph{bond\_length}}{}
Given a set of cartesian coordinates, assign a new particle
a distance of ‘bond\_length’ away in a random direction.

parent\_coordinates: Positions for a single particle,
away from which we will place a new particle a distance
of ‘bond\_length’ away.
( np.array( float * unit.angstrom ( length = 3 ) ) )

bond\_length: Bond length for all beads that are bonded,
( float * simtk.unit.distance )
default = 1.0 * unit.angstrom

trial\_coordinates: Positions for a new trial particle
( np.array( float * unit.angstrom ( length = 3 ) ) )

\end{fulllineitems}

\index{collisions() (in module util)@\spxentry{collisions()}\spxextra{in module util}}

\begin{fulllineitems}
\phantomsection\label{\detokenize{util:util.collisions}}\pysiglinewithargsret{\sphinxcode{\sphinxupquote{util.}}\sphinxbfcode{\sphinxupquote{collisions}}}{\emph{distances}, \emph{bond\_length}}{}
Determine whether there are any collisions between non-bonded
particles, where a “collision” is defined as a distance shorter
than the user-provided ‘bond\_length’.

distances: List of the distances between all nonbonded particles.
( list ( float * simtk.unit.distance ( length = \# nonbonded\_interactions ) ) )

bond\_length: Bond length for all beads that are bonded,
( float * simtk.unit.distance )
default = 1.0 * unit.angstrom

collision: Logical variable stating whether or not the model has
bead collisions.
default = False

\end{fulllineitems}

\index{distance() (in module util)@\spxentry{distance()}\spxextra{in module util}}

\begin{fulllineitems}
\phantomsection\label{\detokenize{util:util.distance}}\pysiglinewithargsret{\sphinxcode{\sphinxupquote{util.}}\sphinxbfcode{\sphinxupquote{distance}}}{\emph{positions\_1}, \emph{positions\_2}}{}
Construct a matrix of the distances between all particles.

positions\_1: Positions for a particle
( np.array( length = 3 ) )

positions\_2: Positions for a particle
( np.array( length = 3 ) )

distance
( float * unit )

\end{fulllineitems}

\index{distance\_matrix() (in module util)@\spxentry{distance\_matrix()}\spxextra{in module util}}

\begin{fulllineitems}
\phantomsection\label{\detokenize{util:util.distance_matrix}}\pysiglinewithargsret{\sphinxcode{\sphinxupquote{util.}}\sphinxbfcode{\sphinxupquote{distance\_matrix}}}{\emph{positions}}{}
Construct a matrix of the distances between all particles.

positions: Positions for an array of particles.
( np.array( num\_particles x 3 ) )

distance\_matrix: Matrix containing the distances between all beads.
( np.array( num\_particles x 3 ) )

\end{fulllineitems}

\index{first\_bead() (in module util)@\spxentry{first\_bead()}\spxextra{in module util}}

\begin{fulllineitems}
\phantomsection\label{\detokenize{util:util.first_bead}}\pysiglinewithargsret{\sphinxcode{\sphinxupquote{util.}}\sphinxbfcode{\sphinxupquote{first\_bead}}}{\emph{positions}}{}
Determine if we have any particles in ‘positions’

positions: Positions for all beads in the coarse-grained model.
( np.array( float * unit ( shape = num\_beads x 3 ) ) )

first\_bead: Logical variable stating if this is the first particle.

\end{fulllineitems}

\index{get\_move() (in module util)@\spxentry{get\_move()}\spxextra{in module util}}

\begin{fulllineitems}
\phantomsection\label{\detokenize{util:util.get_move}}\pysiglinewithargsret{\sphinxcode{\sphinxupquote{util.}}\sphinxbfcode{\sphinxupquote{get\_move}}}{\emph{direction}, \emph{step}}{}
Given the cartesian coordinates for a particle (‘move’),
a ‘step’ (distance), and a ‘direction’ ( Index denoting
x,y,z Cartesian direction), update the coordinates for
the particle.

direction: Cartesian directions in which we have attempted 
a particle placement, where: x=0,y=1,z=2. 
( integer )

step: Number to add/subtract to the cartesian coordinates
for direction ‘direction’ in ‘move’
( float * simtk.unit.distance )

move: Updated positions for the particle
( np.array( float * unit.angstrom ( length = 3 ) ) )

\end{fulllineitems}

\index{non\_bonded\_distances() (in module util)@\spxentry{non\_bonded\_distances()}\spxextra{in module util}}

\begin{fulllineitems}
\phantomsection\label{\detokenize{util:util.non_bonded_distances}}\pysiglinewithargsret{\sphinxcode{\sphinxupquote{util.}}\sphinxbfcode{\sphinxupquote{non\_bonded\_distances}}}{\emph{new\_coordinates}, \emph{existing\_coordinates}}{}
Calculate the distances between a trial particle (‘new\_coordinates’)
and all existing particles (‘existing\_coordinates’).

new\_coordinates: Positions for a single trial particle
( np.array( float * unit.angstrom ( length = 3 ) ) )

existing\_coordinates: Positions for a single trial particle
( np.array( float * unit.angstrom ( shape = num\_particles x 3 ) ) )

distances: List of the distances between all nonbonded particles.
( list ( float * simtk.unit.distance ( length = \# nonbonded\_interactions ) ) )

\end{fulllineitems}

\index{random\_positions() (in module util)@\spxentry{random\_positions()}\spxextra{in module util}}

\begin{fulllineitems}
\phantomsection\label{\detokenize{util:util.random_positions}}\pysiglinewithargsret{\sphinxcode{\sphinxupquote{util.}}\sphinxbfcode{\sphinxupquote{random\_positions}}}{\emph{polymer\_length}, \emph{backbone\_length}, \emph{sidechain\_length}, \emph{sidechain\_positions}, \emph{bond\_length}, \emph{sigma}}{}
Assign random positions for all beads in a coarse-grained polymer.

polymer\_length: Number of monomer units (integer), default = 8

backbone\_length: Number of beads in the backbone 
portion of each (individual) monomer (integer), default = 1

sidechain\_length: Number of beads in the sidechain
portion of each (individual) monomer (integer), default = 1

sidechain\_positions: List of integers defining the backbone
bead indices upon which we will place the sidechains,
default = {[}0{]} (Place a sidechain on the backbone bead with
index “0” (first backbone bead) in each (individual) monomer

bond\_length: Bond length for all beads that are bonded,
( float * simtk.unit.distance )
default = 1.0 * unit.angstrom

sigma: Non-bonded bead Lennard-Jones interaction distances,
( float * simtk.unit.distance )
default = 8.4 * unit.angstrom

bb\_bond\_length: Bond length for all bonded backbone beads,
( float * simtk.unit.distance )
default = 1.0 * unit.angstrom

bs\_bond\_length: Bond length for all backbone-sidechain bonds,
( float * simtk.unit.distance )
default = 1.0 * unit.angstrom

ss\_bond\_length: Bond length for all beads within a sidechain,
( float * simtk.unit.distance )
default = 1.0 * unit.angstrom

positions: Positions for all beads in the coarse-grained model.
( np.array( num\_beads x 3 ) )

\end{fulllineitems}

\index{random\_sign() (in module util)@\spxentry{random\_sign()}\spxextra{in module util}}

\begin{fulllineitems}
\phantomsection\label{\detokenize{util:util.random_sign}}\pysiglinewithargsret{\sphinxcode{\sphinxupquote{util.}}\sphinxbfcode{\sphinxupquote{random\_sign}}}{\emph{number}}{}
Returns ‘number’ with a random sign.

number: float

number

\end{fulllineitems}

\index{unit\_sqrt() (in module util)@\spxentry{unit\_sqrt()}\spxextra{in module util}}

\begin{fulllineitems}
\phantomsection\label{\detokenize{util:util.unit_sqrt}}\pysiglinewithargsret{\sphinxcode{\sphinxupquote{util.}}\sphinxbfcode{\sphinxupquote{unit\_sqrt}}}{\emph{simtk\_quantity}}{}
Returns the square root of a simtk ‘Quantity’.

simtk\_quantity: A ‘Quantity’ object, as defined in simtk.
( float * unit )

sqrt: Square root of a simtk\_quantity.

\end{fulllineitems}

\index{update\_trial\_coordinates() (in module util)@\spxentry{update\_trial\_coordinates()}\spxextra{in module util}}

\begin{fulllineitems}
\phantomsection\label{\detokenize{util:util.update_trial_coordinates}}\pysiglinewithargsret{\sphinxcode{\sphinxupquote{util.}}\sphinxbfcode{\sphinxupquote{update\_trial\_coordinates}}}{\emph{move}, \emph{trial\_coordinates=None}}{}
Updates ‘trial\_coordinates by adding the coordinates in ‘move’.

move: Cartesian coordinates for a particle
( np.array( float * unit ( length = 3 ) ) )

trial\_coordinates: Existing cartesian coordinates for the particle
we are updating.
( np.array( float * unit ( length = 3 ) ) )
Optional, default = None

new\_coordinates: Updated coordinates for the particle.

\end{fulllineitems}



\chapter{Indices and tables}
\label{\detokenize{index:indices-and-tables}}\begin{itemize}
\item {} 
\DUrole{xref,std,std-ref}{genindex}

\item {} 
\DUrole{xref,std,std-ref}{modindex}

\item {} 
\DUrole{xref,std,std-ref}{search}

\end{itemize}


\renewcommand{\indexname}{Python Module Index}
\begin{sphinxtheindex}
\let\bigletter\sphinxstyleindexlettergroup
\bigletter{i}
\item\relax\sphinxstyleindexentry{iotools}\sphinxstyleindexpageref{util:\detokenize{module-iotools}}
\indexspace
\bigletter{u}
\item\relax\sphinxstyleindexentry{util}\sphinxstyleindexpageref{util:\detokenize{module-util}}
\end{sphinxtheindex}

\renewcommand{\indexname}{Index}
\printindex
\end{document}