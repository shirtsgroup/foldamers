%% Generated by Sphinx.
\def\sphinxdocclass{report}
\documentclass[letterpaper,10pt,english]{sphinxmanual}
\ifdefined\pdfpxdimen
   \let\sphinxpxdimen\pdfpxdimen\else\newdimen\sphinxpxdimen
\fi \sphinxpxdimen=.75bp\relax

\PassOptionsToPackage{warn}{textcomp}
\usepackage[utf8]{inputenc}
\ifdefined\DeclareUnicodeCharacter
% support both utf8 and utf8x syntaxes
\edef\sphinxdqmaybe{\ifdefined\DeclareUnicodeCharacterAsOptional\string"\fi}
  \DeclareUnicodeCharacter{\sphinxdqmaybe00A0}{\nobreakspace}
  \DeclareUnicodeCharacter{\sphinxdqmaybe2500}{\sphinxunichar{2500}}
  \DeclareUnicodeCharacter{\sphinxdqmaybe2502}{\sphinxunichar{2502}}
  \DeclareUnicodeCharacter{\sphinxdqmaybe2514}{\sphinxunichar{2514}}
  \DeclareUnicodeCharacter{\sphinxdqmaybe251C}{\sphinxunichar{251C}}
  \DeclareUnicodeCharacter{\sphinxdqmaybe2572}{\textbackslash}
\fi
\usepackage{cmap}
\usepackage[T1]{fontenc}
\usepackage{amsmath,amssymb,amstext}
\usepackage{babel}
\usepackage{times}
\usepackage[Bjarne]{fncychap}
\usepackage{sphinx}

\fvset{fontsize=\small}
\usepackage{geometry}

% Include hyperref last.
\usepackage{hyperref}
% Fix anchor placement for figures with captions.
\usepackage{hypcap}% it must be loaded after hyperref.
% Set up styles of URL: it should be placed after hyperref.
\urlstyle{same}

\addto\captionsenglish{\renewcommand{\figurename}{Fig.\@ }}
\makeatletter
\def\fnum@figure{\figurename\thefigure{}}
\makeatother
\addto\captionsenglish{\renewcommand{\tablename}{Table }}
\makeatletter
\def\fnum@table{\tablename\thetable{}}
\makeatother
\addto\captionsenglish{\renewcommand{\literalblockname}{Listing}}

\addto\captionsenglish{\renewcommand{\literalblockcontinuedname}{continued from previous page}}
\addto\captionsenglish{\renewcommand{\literalblockcontinuesname}{continues on next page}}
\addto\captionsenglish{\renewcommand{\sphinxnonalphabeticalgroupname}{Non-alphabetical}}
\addto\captionsenglish{\renewcommand{\sphinxsymbolsname}{Symbols}}
\addto\captionsenglish{\renewcommand{\sphinxnumbersname}{Numbers}}

\addto\extrasenglish{\def\pageautorefname{page}}

\setcounter{tocdepth}{1}



\title{foldamers Documentation}
\date{Apr 10, 2019}
\release{0.0.1}
\author{Garrett A. Meek, Lenny T. Fobe, Michael R. Shirts}
\newcommand{\sphinxlogo}{\vbox{}}
\renewcommand{\releasename}{Release}
\makeindex
\begin{document}

\pagestyle{empty}
\sphinxmaketitle
\pagestyle{plain}
\sphinxtableofcontents
\pagestyle{normal}
\phantomsection\label{\detokenize{index::doc}}


This documentation is generated automatically using Sphinx, which reads
all docstring-formatted comments from Python functions in
the ‘foldamers’ repository.  (See foldamers/doc for Sphinx
source files.)


\chapter{OpenMM Functions}
\label{\detokenize{openmm:openmm-functions}}\label{\detokenize{openmm::doc}}
This page details the functions and classes in src/openmm.py


\section{‘cgmodel’ class for OpenMM simulation}
\label{\detokenize{openmm:cgmodel-class-for-openmm-simulation}}
Shown below is a detailed description of the ‘cgmodel’ class object,
which contains all simulation objects required by OpenMM.

\phantomsection\label{\detokenize{openmm:module-openmm}}\index{openmm (module)@\spxentry{openmm}\spxextra{module}}\index{cgmodel (class in openmm)@\spxentry{cgmodel}\spxextra{class in openmm}}

\begin{fulllineitems}
\phantomsection\label{\detokenize{openmm:openmm.cgmodel}}\pysiglinewithargsret{\sphinxbfcode{\sphinxupquote{class }}\sphinxcode{\sphinxupquote{openmm.}}\sphinxbfcode{\sphinxupquote{cgmodel}}}{\emph{box\_size=Quantity(value=10.0, unit=nanometer), polymer\_length=12, backbone\_length=1, sidechain\_length=1, sidechain\_positions={[}0{]}, mass=Quantity(value=12.0, unit=dalton), sigma=Quantity(value=8.4, unit=angstrom), epsilon=Quantity(value=0.5, unit=kilocalorie/mole), bond\_length=Quantity(value=1.0, unit=angstrom), bb\_bond\_length=Quantity(value=1.0, unit=angstrom), bs\_bond\_length=Quantity(value=1.0, unit=angstrom), ss\_bond\_length=Quantity(value=1.0, unit=angstrom), charge=Quantity(value=0.0, unit=elementary charge)}}{}
Construct all of the objects that OpenMM expects/requires 
for simulations with a coarse grained model.

box\_size: Simulation box length, 
default = 10.00 * unit.nanometer

polymer\_length: Number of monomer units (integer), default = 8

backbone\_length: Number of beads in the backbone 
portion of each (individual) monomer (integer), default = 1

sidechain\_length: Number of beads in the sidechain
portion of each (individual) monomer (integer), default = 1

sidechain\_positions: List of integers defining the backbone
bead indices upon which we will place the sidechains,
default = {[}0{]} (Place a sidechain on the backbone bead with
index “0” (first backbone bead) in each (individual) monomer

mass: Mass of coarse grained beads ( float * simtk.unit.mass )
default = 12.0 * unit.amu

sigma: Non-bonded bead Lennard-Jones interaction distances,
( float * simtk.unit.distance )
default = 8.4 * unit.angstrom

epsilon: Non-bonded bead Lennard-Jones interaction strength,
( float * simtk.unit.energy )
default = 0.5 * unit.kilocalorie\_per\_mole

bond\_length: Bond length for all beads that are bonded,
( float * simtk.unit.distance )
default = 1.0 * unit.angstrom

bb\_bond\_length: Bond length for all bonded backbone beads,
( float * simtk.unit.distance )
default = 1.0 * unit.angstrom

bs\_bond\_length: Bond length for all backbone-sidechain bonds,
( float * simtk.unit.distance )
default = 1.0 * unit.angstrom

ss\_bond\_length: Bond length for all beads within a sidechain,
( float * simtk.unit.distance )
default = 1.0 * unit.angstrom

charge: Charge for all beads
( float * simtk.unit.charge )
default = 0.0 * unit.elementary\_charge

box\_size
polymer\_length
backbone\_length
sidechain\_length
sidechain\_positions
mass
sigma
epsilon
bond\_length
bb\_bond\_length
bs\_bond\_length
ss\_bond\_length
charge
num\_beads
topology
system
positions
simulation

\end{fulllineitems}



\section{foldamers ‘modules’ for OpenMM simulation}
\label{\detokenize{openmm:module-openmm}}\label{\detokenize{openmm:foldamers-modules-for-openmm-simulation}}\index{openmm (module)@\spxentry{openmm}\spxextra{module}}\index{build\_mm\_force() (in module openmm)@\spxentry{build\_mm\_force()}\spxextra{in module openmm}}

\begin{fulllineitems}
\phantomsection\label{\detokenize{openmm:openmm.build_mm_force}}\pysiglinewithargsret{\sphinxcode{\sphinxupquote{openmm.}}\sphinxbfcode{\sphinxupquote{build\_mm\_force}}}{\emph{sigma}, \emph{epsilon}, \emph{charge}, \emph{num\_beads}, \emph{cutoff=Quantity(value=1}, \emph{unit=nanometer)}}{}
Build an OpenMM ‘Force’ for the non-bonded interactions in our model.

sigma: Non-bonded bead Lennard-Jones interaction distances,
( float * simtk.unit.distance )

epsilon: Non-bonded bead Lennard-Jones interaction strength,
( float * simtk.unit.energy )

charge: Charge for all beads
( float * simtk.unit.charge )

cutoff: Cutoff distance for nonbonded interactions
( float * simtk.unit.distance )

num\_beads: Total number of beads in our coarse grained model
( integer )

\end{fulllineitems}

\index{build\_mm\_simulation() (in module openmm)@\spxentry{build\_mm\_simulation()}\spxextra{in module openmm}}

\begin{fulllineitems}
\phantomsection\label{\detokenize{openmm:openmm.build_mm_simulation}}\pysiglinewithargsret{\sphinxcode{\sphinxupquote{openmm.}}\sphinxbfcode{\sphinxupquote{build\_mm\_simulation}}}{\emph{topology}, \emph{system}, \emph{temperature}, \emph{simulation\_time\_step}, \emph{total\_simulation\_time}, \emph{positions}, \emph{output\_data='output.dat'}, \emph{print\_frequency=100}}{}
Construct an OpenMM simulation object for our coarse grained model.

topology: OpenMM topology object

system: OpenMM system object

temperature: Simulation temperature ( float * simtk.unit.temperature )

simulation\_time\_step: Simulation integration time step
( float * simtk.unit.time )

total\_simulation\_time: Total simulation time ( float * simtk.unit.time )

positions: Array containing the positions of all beads
in the coarse grained model
( np.array( ‘num\_beads’ x 3 , ( float * simtk.unit.distance ) )

output\_data: Name of output file where we will write the data from this
simulation ( string )

print\_frequency: Number of simulation steps to skip when writing data
to ‘output\_data’ ( integer )

\end{fulllineitems}

\index{build\_mm\_system() (in module openmm)@\spxentry{build\_mm\_system()}\spxextra{in module openmm}}

\begin{fulllineitems}
\phantomsection\label{\detokenize{openmm:openmm.build_mm_system}}\pysiglinewithargsret{\sphinxcode{\sphinxupquote{openmm.}}\sphinxbfcode{\sphinxupquote{build\_mm\_system}}}{\emph{box\_size}, \emph{mass}, \emph{num\_beads}, \emph{sigma}, \emph{epsilon}, \emph{charge}}{}
Construct an OpenMM system for our coarse grained model

box\_size: Simulation box length ( float * simtk.unit.length )

mass: Coarse grained particle mass ( float * simtk.unit.length )

num\_beads: Total number of beads in our coarse grained model (int)

sigma: Non-bonded bead Lennard-Jones interaction distances,
( float * simtk.unit.distance )

epsilon: Non-bonded bead Lennard-Jones interaction strength,
( float * simtk.unit.energy )

charge: Charge for all beads
( float * simtk.unit.charge )

\end{fulllineitems}

\index{build\_mm\_topology() (in module openmm)@\spxentry{build\_mm\_topology()}\spxextra{in module openmm}}

\begin{fulllineitems}
\phantomsection\label{\detokenize{openmm:openmm.build_mm_topology}}\pysiglinewithargsret{\sphinxcode{\sphinxupquote{openmm.}}\sphinxbfcode{\sphinxupquote{build\_mm\_topology}}}{\emph{polymer\_length}, \emph{backbone\_length}, \emph{sidechain\_length}}{}
Construct an OpenMM topology for our coarse grained model

polymer\_length: Number of monomers in our coarse grained model
( integer )

backbone\_length: Number of backbone beads on individual monomers
in our coarse grained model, ( integer )

sidechain\_length: Number of sidechain beads on individual monomers
in our coarse grained model, ( integer )

\end{fulllineitems}

\index{get\_box\_vectors() (in module openmm)@\spxentry{get\_box\_vectors()}\spxextra{in module openmm}}

\begin{fulllineitems}
\phantomsection\label{\detokenize{openmm:openmm.get_box_vectors}}\pysiglinewithargsret{\sphinxcode{\sphinxupquote{openmm.}}\sphinxbfcode{\sphinxupquote{get\_box\_vectors}}}{\emph{box\_size}}{}
Assign all side lengths for simulation box.

box\_size: Simulation box length ( float * simtk.unit.length )

\end{fulllineitems}

\index{set\_box\_vectors() (in module openmm)@\spxentry{set\_box\_vectors()}\spxextra{in module openmm}}

\begin{fulllineitems}
\phantomsection\label{\detokenize{openmm:openmm.set_box_vectors}}\pysiglinewithargsret{\sphinxcode{\sphinxupquote{openmm.}}\sphinxbfcode{\sphinxupquote{set\_box\_vectors}}}{\emph{system}, \emph{box\_size}}{}
Build a simulation box.

system: OpenMM system object

box\_size: Simulation box length ( float * simtk.unit.length )

\end{fulllineitems}



\chapter{Indices and tables}
\label{\detokenize{index:indices-and-tables}}\begin{itemize}
\item {} 
\DUrole{xref,std,std-ref}{genindex}

\item {} 
\DUrole{xref,std,std-ref}{modindex}

\item {} 
\DUrole{xref,std,std-ref}{search}

\end{itemize}


\renewcommand{\indexname}{Python Module Index}
\begin{sphinxtheindex}
\let\bigletter\sphinxstyleindexlettergroup
\bigletter{o}
\item\relax\sphinxstyleindexentry{openmm}\sphinxstyleindexpageref{openmm:\detokenize{module-openmm}}
\end{sphinxtheindex}

\renewcommand{\indexname}{Index}
\printindex
\end{document}